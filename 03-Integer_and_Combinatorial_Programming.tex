\part{Integer and Combinatorial Programming}
	\chapter{Formulation}
		\section{Typical Problems}

		\section{Integer Programming Formulation Skills}
			\subsection{A Variable Taking Discontinuous Values}
				\framebox{\textbf{Description:}} In algebraic notation: 
				\begin{equation}
					x = 0,\quad \text{or} \quad l\le x \le u \nonumber
				\end{equation}
				\framebox{\textbf{Modeling:}}
				\begin{align}
					& x \le uy\nonumber \\
					& x \ge ly \nonumber \\
					& y \in \{0, 1\} \nonumber
				\end{align}
				where
				\begin{equation}y=\begin{cases}0, & \text{if }x=0 \\ 1, & \text{if } l\le x \le u\end{cases}\nonumber \end{equation}
					
			\subsection{Fixed Costs}
				\framebox{\textbf{Description:}} In algebraic notation: 
				\begin{equation}
					C(x) = \begin{cases} 0 & \text{for } x=0 \\ k + cx & \text{for } x > 0 \end{cases} \nonumber
				\end{equation}
				\framebox{\textbf{Modeling:}}
				\begin{align}
					& C^*(x, y) = ky+cx\nonumber\\
					& x \le My \nonumber \\
					& x \ge 0 \nonumber\\
					& y \in \{0, 1\} \nonumber
				\end{align}
				where
				\begin{equation}y=\begin{cases}0, & \text{if }x=0 \\ 1, & \text{if }x\ge 0\end{cases}\nonumber \end{equation}
			
			\subsection{Either-or Constraints}
				\framebox{\textbf{Description:}} In algebraic notation: 
				\begin{equation}
					\sum_{j\in J} a_{1j} x_j \le b_1 \text{ or } \sum_{j\in J} a_{2j} x_j \le b_2 \nonumber
				\end{equation}
				\framebox{\textbf{Modeling:}}
				\begin{align}
					& \sum_{j\in J} a_{1j} x_j \le b_1 + M_1y \nonumber \\
					& \sum_{j\in J} a_{2j} x_j \le b_2 + M_1(1-y) \nonumber \\
					& y \in \{0, 1\} \nonumber
				\end{align}
				where
				\begin{equation}y=\begin{cases}0, & \text{if }\sum_{j\in J} a_{1j} x_j \le b_1 \\ 1, & \text{if } \sum_{j\in J} a_{2j} x_j \le b_2\end{cases}\nonumber \end{equation}
				Notice that the sign before $M$ is determined by the inequality $\ge$ or $\le$, if it is \lq\lq{}$\ge$\rq\rq{}, use \lq\lq{}$-$\rq\rq{}, if it \lq\lq{}$\le$\rq\rq{}, use \lq\lq{}+\rq\rq{}.
			
			\subsection{Conditional Constraints}
				\framebox{\textbf{Description:}} If constraint A is satisfied, then constraint B must also be satisfied
				\begin{equation}
					\text{If} \quad \sum_{j\in J} a_{1j} x_j \le b_1 \text{ then } \sum_{j\in J} a_{2j} x_j \le b_2 \nonumber
				\end{equation}
				The key part is to find the opposite of the first condition. We are using $A\Rightarrow B \Leftrightarrow \neg B \Rightarrow \neg A$\\
				Therefore it is equivalent to
				\begin{equation}
					\sum_{j\in J} a_{1j} x_j > b_1 \text{ or } \sum_{j\in J} a_{2j} x_j \le b_2 \nonumber
				\end{equation}
				Furthermore, it is equivalent to
				\begin{equation}
					\sum_{j\in J} a_{1j} x_j \ge b_1 + \epsilon \text{ or } \sum_{j\in J} a_{2j} x_j \le b_2 \nonumber
				\end{equation}
				Where $\epsilon$ is a very small positive number.\\
				\framebox{\textbf{Modeling:}}
				\begin{align}
					& \sum_{j\in J} a_{1j} x_j \ge b_1 + \epsilon -  M_2y \nonumber \\
					& \sum_{j\in J} a_{2j} x_j \le b_2 + M_2(1-y) \nonumber \\
					& y \in \{0, 1\} \nonumber
				\end{align}	
			
			\subsection{Special Ordered Sets}
				\framebox{\textbf{SOS1 Description}} Out of a set of yes-no decisions, at most one decision variable can be yes. \
				\begin{align}
					x_1=1,x_2=x_3&=\dots=x_n=0 \nonumber \\
					&\text{or} \nonumber \\
					x_2=1, x_1=x_3&=\dots=x_n=0 \nonumber \\
					&\text{or ...} \nonumber
				\end{align} 
				\framebox{\textbf{Modeling:}}
				\begin{equation} \sum_{i} x_i = 1, \quad i \in N\nonumber \end{equation}
				\framebox{\textbf{SOS2 Description 1}} Out of a set of binary variables, at most two variables can be nonzero. In addition, the two variables must be adjacent to each other in a fixed order list.\\
				\framebox{\textbf{Modeling:}}
				If $x_1, x_2, ... , x_n$ is a SOS2, then
				\begin{align}
					& \sum_{i=1}^{n} x_i \le 2 \nonumber \\
					& x_i + x_j \le 1, \forall i \in \{1, 2,..., n\}, j \in \{i+2, i+3, ..., n\} \nonumber \\
					&x_i \in \{0, 1\}\nonumber
				\end{align}
				\framebox{\textbf{SOS2 Description 2}} There is another type of definition, that is out of a set of nonnegative variables \textbf{not binary here}, at most two variables can be nonzero. In addition, the two variables must be adjacent to each other in a fixed order list. All variables summing to 1.\\
				This definition of SOS2 is used in the following section \textit{Piecewise Linear Formulations}\\
								
			\subsection{Piecewise Linear Formulations}
				\framebox{\textbf{Description:}} The objective function is a sequence of line segments, e.g. $y=f(x), $ consists $k-1$ linear segments going through $k$ given points $(x_1, y_1), (x_2, y_2), ... ,(x_k, y_k)$.\\
				Denote 
				\begin{equation}d_i=\begin{cases}1, & x\in (x_i, x_{i+1})\\0, & \text{otherwise} \end{cases}\nonumber\end{equation}
				Then the objective function is
				\begin{equation}\sum_{i \in \{1, 2, ..., k-1\}} y = d_if_i(x)\nonumber \end{equation} 
				\framebox{\textbf{Modeling:}} Given that objective function as a piecewise linear formulation, we can have these constraints\\
				\begin{align}
					&\sum_{i \in \{1, 2, ..., k-1\}} d_i =1 \nonumber \\
					&d_i \in \{0, 1\}, i \in \{1, 2, ..., k-1\} \nonumber \\
					& x = \sum_{i \in \{1, 2, ..., k\}} w_i x_i \nonumber \\
					& y = \sum_{i \in \{1, 2, ..., k\}} w_i y_i \nonumber \\
					& w_1 \le d_1 \nonumber \\
					& w_i \le d_{i-1} + d{i}, i \in \{2, 3, ..., k-1\} \nonumber \\
					& w_k \le d_{k-1} \nonumber
				\end{align}
				In this case, $ w_i \in SOS2$ (second definition)		
									
			\subsection{Conditional Binary Variables}
				\framebox{\textbf{Description:}} Choose at most $n$ binary variable to be 1 out of  $x_1, x_2, ... x_m, m\ge n$. If $n=1$ then it is SOS1.\\
				\framebox{\textbf{Modeling:}}
				\begin{equation}
					\sum_{k\in \{1,2,...,m\}} x_k \le n\nonumber
				\end{equation}
				\framebox{\textbf{Description:}} Choose exactly $n$ binary variable to be 1 out of  $x_1, x_2, ... x_m, m\ge n$\\
				\framebox{\textbf{Modeling:}}
				\begin{equation}
					\sum_{k\in \{1,2,...,m\}} x_k = n\nonumber
				\end{equation}
				\framebox{\textbf{Description:}} Choose $x_j$ only if $x_k = 1$\\
				\framebox{\textbf{Modeling:}}
				\begin{equation}x_j = x_k \nonumber \end{equation}
				\framebox{\textbf{Description:}} \lq\lq{}and\rq\rq{} condition, iff $x_1, x_2, ... , x_m =1$ then $y=1$\\
				\framebox{\textbf{Modeling:}}
				\begin{align}
					& y \le x_i, i\in \{1, 2, ..., m\} \nonumber \\
					& y \ge \sum_{i \in \{1, 2, ..., m\}} x_i - (m - 1) \nonumber
				\end{align}

			\subsection{Elimination of Products of Variables}
				\framebox{\textbf{Description:}} For variables $x_1$ and $x_2$,
				\begin{equation}y = x_1 x_2\nonumber\end{equation}
				\framebox{\textbf{Modeling:}} If $x_1, x_2$ are binary, it is the same as \lq\lq{}and\rq\rq{} condition of binary variables.\\
				If $x_1$ is binary, while $x_2$ is continuous and $0 \le x_2 \le u$, then
				\begin{align}
					y &\le ux_1 \nonumber \\
					y &\le x_2 \nonumber \\
					y &\ge x_2 - u(1- x_1) \nonumber \\
					y &\ge 0 \nonumber
				\end{align}
				If both $x_1$ and $x_2$ are continuous, it is non-linear, we can use Piecewise linear formulation to simulate.

	\chapter{Branch and Bound}

	\chapter{Branch and Cut}

	\chapter{Packing and Matching}

	\chapter{Traveling Salesman Problem}

	\chapter{Knapsack Problem}