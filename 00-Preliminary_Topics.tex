\documentclass[10pt, onecolumn]{book}
\author{Lan Peng, PhD Student\\ \\Department of Industrial and Systems Engineering\\University at Buffalo, SUNY\\lanpeng@buffalo.edu}
\title{Notes for Operations Research \& More}

\usepackage{amsmath}
\usepackage{amssymb}
\usepackage{amsfonts}
\usepackage{graphicx}
\usepackage{amsthm}
\usepackage{color}
\usepackage{tabularx}
\usepackage{diagbox}
\usepackage{bm}
\usepackage{mathrsfs}
\usepackage{hyperref}
\usepackage{longtable}
\usepackage{makecell}
\usepackage{lscape}

\usepackage{algorithm}
\usepackage{algpseudocode}
\algtext*{EndWhile}
\algtext*{EndIf}
\algtext*{EndFor}

\usepackage[
	letterpaper,
	left=2cm,
	right=2cm,
	top=2cm,
	bottom=2cm]{geometry}
	\setlength{\parindent}{0pt}

\usepackage{subcaption}
\usepackage{tikz}
	\usetikzlibrary{chains, arrows,shapes,matrix}
	\usetikzlibrary{decorations.pathmorphing} 
	\usepgflibrary{plotmarks}
	\usetikzlibrary{patterns}  
	\usetikzlibrary{positioning} 
	\tikzstyle{roundedRectangle} = [
		rectangle, 
		rounded corners, 
		minimum width=3cm, 
		minimum height=1cm, 
		text centered, 
		draw=black
	]
	\tikzstyle{io} = [
		trapezium, 
		trapezium left angle=70, 
		trapezium right angle=110, 
		minimum width=3cm, 
		minimum height=1cm, 
		text centered, 
		draw=black
	]
	\tikzstyle{process} = [
		rectangle, 
		minimum width=2cm, 
		minimum height=1cm, 
		text centered, 
		draw=black, 
		inner sep=0.1cm
	]
	\tikzstyle{decision} = [
		diamond, 
		minimum width=2cm, 
		minimum height=0cm, 
		text centered, 
		draw=black, 
		inner sep=0cm
	]
	\tikzstyle{arrow} = [
		thick,
		->,
		>=stealth
	]
	\tikzstyle{link} = [
		thick, 
		-
	]
	\tikzstyle{matchedLink} = [
		decorate, 
		decoration={snake}
	]
	\tikzstyle{circleNode} = [
		circle, 
		minimum size = 0.7cm, 
		text centered, 
		draw=black, 
		inner sep=0.1cm
	]
	\tikzstyle{solidNode} = [
		circle, 
		minimum size = 0.1cm, 
		fill=black
	]
	\tikzstyle{smallSolidNode} = [
		circle, 
		minimum size = 0.03cm, 
		fill=black
	]
	\tikzstyle{rectangleCell} = [
		rectangle, 
		minimum width=0.8cm, 
		text centered, 
		draw=black
	]
	\tikzstyle{rowArray} = [
		matrix of nodes, 
		nodes = {draw}, 
		row 1/.style = {
			nodes = {
				draw = none
			}
		}
	]
	\tikzstyle{colArray} = [
		matrix of nodes, 
		nodes = {draw}, 
		column 2/.style = {
			nodes = {
				draw = none
			}
		}
	]


\theoremstyle{definition}
	\newtheorem{definition}{Definition}[section]
	\newtheorem*{example}{Example}
	\newtheorem{problem}{Problem}[chapter]
	\newtheorem*{solution}{Solution}
	\newtheorem{hypothesis}{Hypothesis}[section]
\theoremstyle{plain}
	\newtheorem{theorem}{Theorem}[chapter]
	\newtheorem{corollary}{Corollary}[theorem]
	\newtheorem{lemma}[theorem]{Lemma}
	\newtheorem{conjecture}{Conjecture}
	\newtheorem{proposition}{Proposition}
\theoremstyle{remark}
	\newtheorem*{remark}{Remark}

\usepackage[square,numbers]{natbib}
	\bibliographystyle{plainnat}
	\bibpunct[, ]{(}{)}{,}{a}{}{,}
	\def\bibfont{\small}
	\def\bibsep{\smallskipamount}
	\def\bibhang{24pt}
	\def\newblock{\ }
	\def\BIBand{and}

\newcommand{\todo}[1]{
	\vspace{5 mm}
	\par
	\noindent
	\marginpar{\textsc{to do}}
	\framebox{
		\begin{minipage}[c]{0.95 \textwidth}
		\tt
		\begin{center} 
			#1
		\end{center}
		\end{minipage}
	}
	\vspace{5 mm}
	\par
}

\newcommand{\notice}[1]{
	\vspace{2 mm}
	\par
	\noindent
	\colorbox{gray!15}{
		\centering
		\begin{minipage}[c]{1 \textwidth}
			\textbf{Notice:}~#1
		\end{minipage}
	}
	\vspace{2 mm}
	\par
}

\newcommand{\fixme}[1]{
	{\color{red}#1}
	\marginpar{
		\textsc{
			\color{red}
			FIXME
		}
	}
}
\begin{document}
\part{Preliminary Topics}
	\chapter{Introduction to Optimization}
		\section{Optimization Model}
			The following is the basic forms of terminology:
				\begin{align}
					\text{(P)} \quad \text{min} \quad & f(x)  \\
								\text{s.t.} \quad & g_i(x)\le 0, \quad i=1,2,...,m \\
											& h_j(x)=0, \quad j=1,2,...,l \\
											& x \in X 
				\end{align}
				We have
				\begin{itemize}
					\item - $x\in R^n \rightarrow X \subseteq R^m$
					\item $g_i(x)$ are called inequality constraints
					\item $h_j(x)$ are called equality constraints
					\item $X$ is the domain of the variables (e.g. cone, polygon, $\{0, 1\}^n$, etc.)
					\item Let $F$ be the feasible region of $(P)$:
					\begin{itemize}
						\item $x^0$ is a feasible solution iff $x^0\in F$
						\item $x^*$ is an optimized solution iff $x^* \in F$ and $f(x^*)\le f(x^0), \forall x^0 \in F$ (for minimized problem)
					\end{itemize}
				\end{itemize}

			\notice{Not every $(P)$ has a feasible region, we can have $F=\emptyset$. Even if $F \ne \emptyset$, there might not be an solution to $P$, e.g. unbounded. If $(P)$ has optimized solution(s), it could be 1) Unique 2) Infinite number of solution 3) Finite number of solution}

			Types of Optimization Problem
			\begin{itemize}
				\item $m = l = 0$, $x \in R^n$, unconstrained problem
				\item $m + l > 0$, constrained problem
				\item $f(x), g_i(x), h_j(x)$ are linear, Linear Optimization
				\begin{itemize}
					\item If $X=R^n$, Linear Programming
					\item If $X$ is discrete,  Discrete Optimization
					\item If $X \subseteq Z^n$, Integer Programming
					\item If $X\in \{0,1\}^n$, Binary Programming
					\item If $X\in Z^n \times R^m$, Mixed Integer Programming
				\end{itemize}
			\end{itemize}

		\section{Problem Manipulation}
			\subsection{Inequalities and Equalities}
				An inequality can be transformed into an equation by adding or subtracting the nonnegative slack or surplus variable
				\begin{equation}
					\sum_{j=1}^na_{ij}x_j \ge b_i \Rightarrow \sum_{j=1}^na_{ij}x_j - x_{n+1} = b_i 
				\end{equation}
				or
				\begin{equation}
					\sum_{j=1}^na_{ij}x_j \le b_i \Rightarrow \sum_{j=1}^na_{ij}x_j + x_{n+1} = b_i 
				\end{equation}
				Although it is not the practice, equality can be transformed into inequality too
				\begin{equation}
					\sum_{j=1}^na_{ij}x_j = b_i \Rightarrow \begin{cases}\sum_{j=1}^na_{ij}x_j \le b_i \\ \sum_{j=1}^na_{ij}x_j \ge b_i \end{cases} 
				\end{equation}
				Also, in linear programming, we only care about close set, so we will not have $<$, $>$ in the formulation, we can use the following
				\begin{equation}
					\sum_{j=1}^na_{ij}x_j > b_i \Rightarrow \sum_{j=1}^na_{ij}x_j \ge b_i + \epsilon 
				\end{equation}
				where $\epsilon$ is a small number.

			\subsection{Minimization and Maximization}
				To convert a minimization problem into a maximization problem, we can use the following to define a new objective function
				\begin{equation}
					\min \sum_{j=1}^nc_jx_j = -\max \sum_{j=1}^n c_jx_j 
				\end{equation}

			\subsection{Standard Form and Canonical Form}
				Standard Form
				\begin{align}
					\min \quad & \sum_{j=1}^nc_jx_j \\
					\text{s.t.} \quad & \mathbf{Ax} = \mathbf{b} \\
									  & \mathbf{x} \ge \mathbf{0} 
				\end{align}
				Canonical Form
				\begin{align}
					\min \quad & \sum_{j=1}^nc_jx_j \\
					\text{s.t.} \quad & \mathbf{Ax} \le \mathbf{b} \\
									  & \mathbf{x} \ge \mathbf{0}
				\end{align}

		\section{Typical Linear Programming Problems}

		\section{Linear Programming Formulation Skills}
			\subsection{Absolute Value}
				Consider the following model statement:
				\begin{align}
					\min \quad & \sum_{j\in J}c_j|x_j|, \quad c_j > 0 \\
					\text{s.t.} \quad & \sum_{j\in J}a_{ij}x_j \gtreqless b_i, \quad \forall i\in I \\
					                  & x_j \quad \text{unrestricted}, \quad \forall j\in J 
				\end{align}
				Modeling:
				\begin{align}
					\min \quad & \sum_{j\in J}c_j(x_j^+ + x_j^-), \quad c_j > 0 \\
					\text{s.t.} \quad & \sum_{j\in J}a_{ij}(x_j^+ - x_j^-) \gtreqless b_i, \quad \forall i\in I \\
					                  & x_j^+, x_j^- \ge 0, \quad \forall j\in J 
				\end{align}

			\subsection{A Minimax Objective}
				Consider the following model statement:
				\begin{align}
					\min \quad & \max_{k\in K}\sum_{j\in J}c_{kj}x_j \\
					\text{s.t.} \quad & \sum_{j\in J}a_{ij}x_j \gtreqless b_i, \quad \forall i\in I \\
					                  & x_j \ge 0, \quad \forall j\in J 
				\end{align}
				Modeling:
				\begin{align}
					\min \quad & z \\
					\text{s.t.} \quad & \sum_{j\in J}a_{ij}x_j \gtreqless b_i, \quad \forall i\in I \\
									  & \sum_{j\in J}c_{kj}x_j \le z, \quad \forall k\in K \\
					                  & x_j \ge 0, \quad \forall j\in J 
				\end{align}
			\subsection{A Fractional Objective}
				Consider the following model statement:
				\begin{align}
					\min \quad & \frac{\sum_{j\in J}c_{j}x_j + \alpha}{\sum_{j\in J}d_{j}x_j + \beta} \\
					\text{s.t.} \quad & \sum_{j\in J}a_{ij}x_j \gtreqless b_i, \quad \forall i\in I \\
					                  & x_j \ge 0, \quad \forall j\in J 
				\end{align}
				Modeling:
				\begin{align}
					\min \quad & \sum_{j\in J}c_{j}x_jt + \alpha t \\
					\text{s.t.} \quad & \sum_{j\in J}a_{ij}x_j \gtreqless b_i, \quad \forall i\in J \\
									  & \sum_{j\in J}d_jx_jt + \beta t = 1\\
									  & t > 0 \\
					                  & x_j \ge 0, \quad \forall j\in J \\
					                  & (t = \frac1{\sum_{j\in J}d_jx_j + \beta}) 
				\end{align}

				For the following statement:
				\begin{align}
					\min \quad & z^P = \frac{\mathbf{c^\top x} + d}{\mathbf{e^\top x} + f}\\
					\text{s.t.} \quad & \mathbf{Gx} \le \mathbf{h}\\
					                  & \mathbf{Ax} = \mathbf{b}
				\end{align}

				Modeling
				\begin{align}
					\min \quad & z^R = \mathbf{c^\top y} + dz\\
					\text{s.t.} \quad & \mathbf{Gy} - \mathbf{h}z \le 0\\
									  & \mathbf{Ay} - \mathbf{b}z = 0\\
									  & \mathbf{e^\top y} + fz = 1\\
									  & z \ge 0
				\end{align}

			\subsection{A Range Constraint}
				Consider the following model statement:
				\begin{align}
					\min \quad & \sum_{j\in J}c_jx_j \\
					\text{s.t.} \quad & d_i\le \sum_{j\in J}a_{ij}x_j \le e_i, \quad \forall i\in I \\
					                  & x_j \ge 0, \quad \forall j\in J 
				\end{align}
				Modeling:
				\begin{align}
					\min \quad & \sum_{j\in J}c_jx_j, \quad c_j > 0 \\
					\text{s.t.} \quad & u_i + \sum_{j\in J}a_{ij}x_j = e_i, \quad \forall i\in I \\
					                  & x_j \ge 0, \quad \forall j\in J \\
					                  & 0\le u_i \le e_i-d_i, \quad \forall i\in I 
				\end{align}

		\section{Typical Integer Programming Problems}

		\section{Integer Programming Formulation Skills}
			\subsection{A Variable Taking Discontinuous Values}
				In algebraic notation: 
				\begin{equation}
					x = 0,\quad \text{or} \quad l\le x \le u 
				\end{equation}
				Modeling:
				\begin{align}
					& x \le uy \\
					& x \ge ly  \\
					& y \in \{0, 1\} 
				\end{align}
				where
				\begin{equation}y=\begin{cases}0, & \text{if }x=0 \\ 1, & \text{if } l\le x \le u\end{cases} \end{equation}

			\subsection{Fixed Costs}
				In algebraic notation: 
				\begin{equation}
					C(x) = \begin{cases} 0 & \text{for } x=0 \\ k + cx & \text{for } x > 0 \end{cases} 
				\end{equation}
				Modeling:
				\begin{align}
					& C^*(x, y) = ky+cx\\
					& x \le My  \\
					& x \ge 0 \\
					& y \in \{0, 1\} 
				\end{align}
				where
				\begin{equation}y=\begin{cases}0, & \text{if }x=0 \\ 1, & \text{if }x\ge 0\end{cases} \end{equation}

			\subsection{Either-or Constraints}
				In algebraic notation: 
				\begin{equation}
					\sum_{j\in J} a_{1j} x_j \le b_1 \text{ or } \sum_{j\in J} a_{2j} x_j \le b_2 
				\end{equation}
				Modeling:
				\begin{align}
					& \sum_{j\in J} a_{1j} x_j \le b_1 + M_1y  \\
					& \sum_{j\in J} a_{2j} x_j \le b_2 + M_1(1-y)  \\
					& y \in \{0, 1\} 
				\end{align}
				where
				\begin{equation}y=\begin{cases}0, & \text{if }\sum_{j\in J} a_{1j} x_j \le b_1 \\ 1, & \text{if } \sum_{j\in J} a_{2j} x_j \le b_2\end{cases} \end{equation}
				Notice that the sign before $M$ is determined by the inequality $\ge$ or $\le$, if it is \lq\lq{}$\ge$\rq\rq{}, use \lq\lq{}$-$\rq\rq{}, if it \lq\lq{}$\le$\rq\rq{}, use \lq\lq{}+\rq\rq{}.

			\subsection{Conditional Constraints}
				If constraint A is satisfied, then constraint B must also be satisfied
				\begin{equation}
					\text{If} \quad \sum_{j\in J} a_{1j} x_j \le b_1 \text{ then } \sum_{j\in J} a_{2j} x_j \le b_2 
				\end{equation}
				The key part is to find the opposite of the first condition. We are using $A\Rightarrow B \Leftrightarrow \neg B \Rightarrow \neg A$\\
				Therefore it is equivalent to
				\begin{equation}
					\sum_{j\in J} a_{1j} x_j > b_1 \text{ or } \sum_{j\in J} a_{2j} x_j \le b_2 
				\end{equation}
				Furthermore, it is equivalent to
				\begin{equation}
					\sum_{j\in J} a_{1j} x_j \ge b_1 + \epsilon \text{ or } \sum_{j\in J} a_{2j} x_j \le b_2 
				\end{equation}
				Where $\epsilon$ is a very small positive number.\\
				Modeling:
				\begin{align}
					& \sum_{j\in J} a_{1j} x_j \ge b_1 + \epsilon -  M_2y  \\
					& \sum_{j\in J} a_{2j} x_j \le b_2 + M_2(1-y)  \\
					& y \in \{0, 1\} 
				\end{align}

			\subsection{Special Ordered Sets}
				Out of a set of yes-no decisions, at most one decision variable can be yes. Also known as SOS1.
				\begin{align}
					x_1=1,x_2=x_3&=\dots=x_n=0  \\
					&\text{or}  \\
					x_2=1, x_1=x_3&=\dots=x_n=0  \\
					&\text{or ...} 
				\end{align} 
				Modeling:
				\begin{equation} \sum_{i} x_i = 1, \quad i \in N \end{equation}
				Out of a set of binary variables, at most two variables can be nonzero. In addition, the two variables must be adjacent to each other in a fixed order list. Also known as SOS2.
				Modeling:
				If $x_1, x_2, ... , x_n$ is a SOS2, then
				\begin{align}
					& \sum_{i=1}^{n} x_i \le 2  \\
					& x_i + x_j \le 1, \forall i \in \{1, 2,..., n\}, j \in \{i+2, i+3, ..., n\}  \\
					&x_i \in \{0, 1\}
				\end{align}
				There is another type of definition, that is out of a set of nonnegative variables \textbf{not binary here}, at most two variables can be nonzero. In addition, the two variables must be adjacent to each other in a fixed order list. All variables summing to 1.\\
				This definition of SOS2 is used in Piecewise Linear Formulations.

			\subsection{Piecewise Linear Formulations}
				The objective function is a sequence of line segments, e.g. $y=f(x), $ consists $k-1$ linear segments going through $k$ given points $(x_1, y_1), (x_2, y_2), ... ,(x_k, y_k)$.\\
				Denote 
				\begin{equation}d_i=\begin{cases}1, & x\in (x_i, x_{i+1})\\0, & \text{otherwise} \end{cases}\end{equation}
				Then the objective function is
				\begin{equation}\sum_{i \in \{1, 2, ..., k-1\}} y = d_if_i(x) \end{equation} 
				Modeling: Given that objective function as a piecewise linear formulation, we can have these constraints\\
				\begin{align}
					&\sum_{i \in \{1, 2, ..., k-1\}} d_i =1  \\
					&d_i \in \{0, 1\}, i \in \{1, 2, ..., k-1\}  \\
					& x = \sum_{i \in \{1, 2, ..., k\}} w_i x_i  \\
					& y = \sum_{i \in \{1, 2, ..., k\}} w_i y_i  \\
					& w_1 \le d_1  \\
					& w_i \le d_{i-1} + d{i}, i \in \{2, 3, ..., k-1\}  \\
					& w_k \le d_{k-1} 
				\end{align}
				In this case, $ w_i \in SOS2$ (second definition)

			\subsection{Conditional Binary Variables}
				Choose at most $n$ binary variable to be 1 out of  $x_1, x_2, ... x_m, m\ge n$. If $n=1$ then it is SOS1.\\
				Modeling:
				\begin{equation}
					\sum_{k\in \{1,2,...,m\}} x_k \le n
				\end{equation}
				Choose exactly $n$ binary variable to be 1 out of  $x_1, x_2, ... x_m, m\ge n$\\
				Modeling:
				\begin{equation}
					\sum_{k\in \{1,2,...,m\}} x_k = n
				\end{equation}
				Choose $x_j$ only if $x_k = 1$\\
				Modeling:
				\begin{equation}x_j = x_k  \end{equation}
				\lq\lq{}and\rq\rq{} condition, iff $x_1, x_2, ... , x_m =1$ then $y=1$\\
				Modeling:
				\begin{align}
					& y \le x_i, i\in \{1, 2, ..., m\}  \\
					& y \ge \sum_{i \in \{1, 2, ..., m\}} x_i - (m - 1) 
				\end{align}

			\subsection{Elimination of Products of Variables}
				For variables $x_1$ and $x_2$,
				\begin{equation}y = x_1 x_2\end{equation}
				Modeling: If $x_1, x_2$ are binary, it is the same as \lq\lq{}and\rq\rq{} condition of binary variables.\\
				If $x_1$ is binary, while $x_2$ is continuous and $0 \le x_2 \le u$, then
				\begin{align}
					y &\le ux_1  \\
					y &\le x_2  \\
					y &\ge x_2 - u(1- x_1)  \\
					y &\ge 0 
				\end{align}
				If both $x_1$ and $x_2$ are continuous, it is non-linear, we can use Piecewise linear formulation to simulate.

		\section{Typical Nonlinear Programming Problems}

		\section{Nonlinear Programming Formulation Skills}

	\chapter{Review of Linear Algebra}
		\section{Field}
			\begin{definition}[Field]
				Let $F$ denote either the set  of real numbers or the set of complex numbers.
				\begin{itemize}
					\item Addition is commutative: $x + y = y + x, \forall x, y \in F$
					\item Addition is associative: $x + (y + z) = (x + y) + z, \forall x, y, z \in F$
					\item Element 0 exists and unique: $\exists 0, x + 0 = x, \forall x \in F$
					\item To each $x \in F$ there corresponds a unique element $(-x) \in F$ such that $x + (-x) = 0$
					\item Multiplication is commutative: $xy = yx, \forall x, y \in F$
					\item Multiplication is associative: $x(yz) = (xy)z, \forall x, y, z \in F$
					\item Element 1 exists and unique: $\exists 1, x1=x, \forall x \in F$
					\item To each $x\neq 0 \in F$ there corresponds a unique element $x^{-1} \in F$ that $xx^{-1} = 1$
					\item Multiplication distributes over addition: $x(y + z) = xy + xz, \forall x, y, z \in F$
				\end{itemize}
				Suppose one has a set $F$ of objects $x, y, z, ...$ and two operations on the elements of $F$ as follows. The first operation, called addition, associates with each pair of elements $x, y \in F$ an element $(x + y)\in F$; the second operation, called multiplication, associates with each pair $x, y$ an element $xy \in F$; and these two operations satisfy all conditions above. The set $F$, together with these two operations, is then called a \textbf{field}.
			\end{definition}

			\begin{definition}[Subfield]
				A \textbf{subfield} of the field $C$ is a set $F$ of complex numbers which itself is a field.
			\end{definition}

			\begin{example}
				The set of integers is not a field.
			\end{example}

			\begin{example}
				The set of rational numbers is a field.
			\end{example}

			\begin{example}
				The set of all complex numbers of the form $x + y\sqrt{2}$ where $x$ and $y$ are rational, is a subfield of $\mathbb{C}$.
			\end{example}

			\notice{In this note, we (...Lan) assume that the field involved is a subfield of the complex numbers $\mathbb{C}$. More generally, if $F$ is a field, it may be possible to add the unit $1$ to itself a finite number of times and obtain $0$, which does not happen in the subfield of $\mathbb{C}$. If it does happen in $F$, the least $n$ such that the sum of $n$ 1's is 0 is called \textbf{characteristic} of the field $F$. If it does not happen, then $F$ is called a field of \textbf{characteristic zero}.}

		\section{Real Vector Spaces}

		\section{Linear, Conic, Affine, and Convex Combinations}

		\section{Determinants}

		\section{Inner Products}
			\begin{definition}[Inner Product]
				Let $F$ be the field of real numbers or the field of complex numbers, and $V$ a vector space over $F$. An \textbf{inner product} on $V$ is a function which assigns to each ordered pair of vectors $\mathbf{\alpha}$, $\mathbf{\beta}$ in $V$ a scalar $<\mathbf{\alpha}|\mathbf{\beta}>$ in $F$ in such a way that $\forall \mathbf{\alpha}, \mathbf{\beta}, \mathbf{\gamma} \in V, c \in \mathbb{R}$ that
				\begin{itemize}
					\item $<\mathbf{\alpha}+\mathbf{\beta}|\mathbf{\gamma}> = <\mathbf{\alpha}|\mathbf{\gamma}> + <\mathbf{\beta}|\mathbf{\gamma}>$
					\item $<c\mathbf{\alpha}|\mathbf{\beta}> = c<\mathbf{\alpha}|\mathbf{\beta}>$
					\item $<\mathbf{\alpha}|\mathbf{\beta}> = \overline{<\mathbf{\beta}|\mathbf{\alpha}>}$
					\item $<\mathbf{\alpha}|\mathbf{\alpha}> \ge 0$, $<\mathbf{\alpha}|\mathbf{\alpha}> = 0$ iff $\mathbf{\alpha} = \mathbf{0}$
				\end{itemize}
				Furthermore, the above properties imply that
				\begin{itemize}
					\item $<\mathbf{\alpha}|c\mathbf{\beta}+\mathbf{\gamma}> = \bar{c}<\mathbf{\alpha}|\mathbf{\beta}> + <\mathbf{\alpha}|\mathbf{\gamma}>$
				\end{itemize}
			\end{definition}

			\begin{definition}
				On $F^n$ there is an inner product which we call the \textbf{standard inner product}. It is defined on $\mathbf{\alpha} = (x_1, x_2, ..., x_n)$ and $\mathbf{\beta} = (y_1, y_2, ..., y_n)$ by
				\begin{equation}
					<\mathbf{\alpha}|\mathbf{\beta}> = \sum_j x_j \bar{y_j}
				\end{equation}
				For $F = \mathbb{R}^n$
				\begin{equation}
					<\mathbf{\alpha}|\mathbf{\beta}> = \sum_j x_j y_j
				\end{equation}
				In the real case, the standard inner product is often called the dot product and denoted by $\mathbf{\alpha} \cdot \mathbf{\beta}$
			\end{definition}

			\begin{example}
				For $\mathbf{\alpha} = (x_1, x_2)$ and $\mathbf{\beta} = (y_1, y_2)$ in $\mathbb{R}^2$, the following is an inner product.
				\begin{equation}
					<\mathbf{\alpha}|\mathbf{\beta}> = x_1y_1 - x_2y_1 - x_1y_2 + 4x_2y_2
				\end{equation}
			\end{example}

			\begin{example}
				For $\mathbb{C}^{n\times n}$, 
				\begin{equation}
					<\mathbf{A}|\mathbf{B}> = trace(\mathbf{B}^* \mathbf{A})
				\end{equation}
				is an inner product, where
				\begin{equation}
					\mathbf{A}^*_{ij} = \bar{\mathbf{A}}_{ji} \quad (\textbf{conjugate transpose})
				\end{equation}
				For $\mathbb{R}^{n\times n}$,
				\begin{equation}
					<\mathbf{A}|\mathbf{B}> = trace(\mathbf{B}^T \mathbf{A}) = \sum_j (AB^T)_{jj} = \sum_j\sum_k A_{jk}B_{jk}
				\end{equation}
			\end{example}

		\section{Norms}
			\begin{definition}[Norms]
				A \textbf{norm} on a vector space $\mathcal{V}$ is a function $\|\cdot\|:\mathcal{V} \rightarrow \mathbb{R}$ for which the following three properties hold for all point $\mathbf{x}, \mathbf{y} \in \mathcal{V}$ and scalars $\lambda \in \mathbb{R}$
				\begin{itemize}
					\item (Absolute homogeneity) $\|\lambda \mathbf{x} \| = |\lambda| \|\mathbf{x}\|$
					\item (Triangle inequality) $\|\mathbf{x} + \mathbf{y}\| \le \|\mathbf{x}\| + \|\mathbf{y}\|$
					\item (Positivity) Equality $\|\mathbf{x}\| = 0$ holds iff $\mathbf{x} = 0$
				\end{itemize}
			\end{definition}

			\begin{definition}[$L_p$-norms]
				Let $p \ge 1$ be a real number. We define the $p$-norm of vector $\mathbf{v}\in \mathbb{R}^n$ as:
				\begin{equation}
					\|\mathbf{x}\|_p = (\sum_{i = 1}^n|v_i|^p)^{\frac{1}{p}}
				\end{equation}
				Particularly
				\begin{align}
					\|\mathbf{v}\|_1 &= \sum_{i = 1}^n|v_i|\\
					\|\mathbf{v}\|_2 &= \sqrt{\sum_{i = 1}^n v_i^2}\\
					\|\mathbf{v}\|_\infty &= \max_{i=1}^n |v_i|
				\end{align}
			\end{definition}

			\begin{definition}[Frobenius norm]
				$\mathbf{X} \in \mathbb{R}^{m \times n}$, the \textbf{Frobenius norm} is defined as
				\begin{equation}
					\|\mathbf{X}\|_F = \sqrt{trace(\mathbf{X}^\top \mathbf{X})}
				\end{equation}
			\end{definition}

			\begin{definition}[Dual norm]
				For an arbitrary norm $\|\cdot\|$ on Euclidean space $\mathbf{E}$, the \textbf{dual norm} $\|\cdot\|^*$ on $\mathbf{E}$ is defined by
				\begin{equation}
					\|\mathbf{v}\|^* = \max \{<\mathbf{v}|\mathbf{x}>|\|\mathbf{x}\| \le 1\}
				\end{equation}
			\end{definition}

			For $p, q \in [1, \infty]$, the $l_p$ and $l_q$ norms on $\mathbb{R}^n$ are dual to each other whenever $\frac1p + \frac1q = 1$.

		\section{Eigenvectors and Eigenvalues}
			\begin{definition}
				If $\mathbf{A}$ is an $n \times n$ matrix, then a nonzero vector $\mathbf{x} \in \mathbb{R}^n$ is called an \textbf{eigenvector} of $\mathbf{A}$ if $\mathbf{Ax}$ is a scalar multiple of $\mathbf{x}$, i.e.
				\begin{equation}
					\mathbf{Ax} = \lambda \mathbf{x}
				\end{equation}
				for some scalar $\lambda$. The scalar $\lambda$ is called \textbf{eigenvalue} of $\mathbf{A}$ and the vector $\mathbf{x}$ is said to be an \textbf{eigenvector corresponding to $\lambda$}
			\end{definition}

			\begin{theorem}[Characteristic Equation]
				If $\mathbf{A}$ is an $n \times n$ matrix, then $\lambda$ is an eigenvalue of $\mathbf{A}$ iff
				\begin{equation}
					\det(\lambda I - A) = 0
				\end{equation}
			\end{theorem}

			\begin{corollary}
				\begin{equation}
					\sum \lambda_A = tr(\mathbf{A})
				\end{equation}
			\end{corollary}

			\begin{corollary}
				\begin{equation}
					\prod \lambda_A = \det(\mathbf{A})
				\end{equation}
			\end{corollary}

			\notice{Gaussian elimination changes the eigenvalues.}

		\section{Decompositions}

	\chapter{Review of Real Analysis}
		\section{The Real Number System}

		\section{Open Sets and Closed Sets}
			\begin{definition}[Metric space]
				A \textbf{metric space} is a set $X$ where we have a notion of distance. That is, if $x, y \in X$, then $d(x, y)$ is the distance between $x$ and $y$. The particular distance function must satisfy the following conditions:
				\begin{itemize}
					\item $d(x, y) > 0, \forall x, y \in X$
					\item $d(x, y) = 0 \iff x=y$
					\item $d(x, y) = d(y, x)$
					\item $d(x, z) \le d(x, y) + d(y, z)$
				\end{itemize}
			\end{definition}

			\begin{definition}[Ball]
				Let $X$ be a metric space. A \textbf{ball} $B$ of radius $r$ around a point $x \in X$ is
				\begin{equation}
					B = \{y \in X|d(x, y) < r\}
				\end{equation}
			\end{definition}

			\begin{definition}[Open set]
				A subset $O \subseteq X$ is \textbf{open} if $\forall x \in O, \exists r, B=\{x\in X|d(x, y) < r\} \subseteq O$
			\end{definition}

			\begin{theorem}
				The union of any collection if open sets is open.
			\end{theorem}

			\begin{proof}
				Sets $S_1, S_2, ..., S_n$ are open sets, let $S = \cup_{i=1}^n S_i$, then $\forall i, S_i \subseteq S$. $\forall x\in S, \exists i, x\in S_i$. Given that $S_i$ is an open set, then for $x$, $\exists r$ that $B = \{x \in S_i | d(x, y) < r\} \subseteq S_i \subseteq S$, therefore $S$ is an open set.
			\end{proof}

			\begin{theorem}
				The intersection of any finite number of open sets is open.
			\end{theorem}

			\begin{proof}
				Sets $S_1, S_2, ..., S_n$ are open sets, let $S = \cap_{i=1}^n S_i$, then $\forall i, S \subseteq S_i$. $\forall x\in S, x\in S_i$. For any $i$, we can define an $r_i$, such that $B_i = \{x\in S_i|d(x, y) < r_i\} \subseteq S_i$. Let $r = \min_i\{r_i\}$. Noticed that $\forall i, B^\prime = \{x \in S_i|d(x, y) < r\} \subseteq B_i \subseteq S_i$. Therefore $S$ is an open set.
			\end{proof}

			\begin{remark}
				The intersection of infinite number of open sets is not necessarily open.
			\end{remark}

			Here we find an example that the intersection of infinite number of open sets can be closed.
			\begin{example}
				Let $A_n \in \mathbb{R}$ and $B_n \in \mathbb{R}$ be two infinite series, with the following properties.
				\begin{itemize}
					\item $\forall n, A_n < a, \lim A_n = a$
					\item $\forall n, B_n > b, \lim B_n = b$
					\item $a < b$
				\end{itemize}
				Then we define infinite number of sets $S_i$, the $i$th set is defined as
				\begin{equation}
					S_i = (A_i, B_i) \subset \mathbb{R}
				\end{equation}
				Then
				\begin{equation}
					S = \cap_{i=1}^\infty S_i = [a, b] \subset \mathbb{R}
				\end{equation}
				and $S$ is a closed set.
			\end{example}

			\begin{definition}[Limit point]
				A point $z$ is a \textbf{limit point} for a set $A$ if every open set $U$ that $z\in U$ intersects $A$ in a point other than $z$.
			\end{definition}

			\notice{$z$ is not necessarily in $A$.}

			\begin{definition}[Closed set]
				A set $C$ is \textbf{closed} iff it contains all of its limit points.
			\end{definition}

			\begin{theorem}
				$S\in \mathbb{R}^n$ is closed $\iff \forall \{x_k\}_{k=1}^\infty \in S, \lim_{k \rightarrow \infty} \{x_k\}_{k=1}^\infty \in S$
			\end{theorem}

			\begin{theorem}
				Every intersection of closed sets is closed.
			\end{theorem}

			\begin{theorem}
				Every finite union of closed sets is closed.
			\end{theorem}

			\begin{remark}
				The union of infinite number of closed sets is not necessarily closed.
			\end{remark}

			\begin{theorem}
				A set $C$ is a closed set if $X \setminus C$ is open
			\end{theorem}

			\begin{proof}
				Let $S$ be an open set, $x \notin S$, for any open set $S_i$ that $x\in S_i$, we can find a correspond $r_i > 0$, such that $B_i = \{x \in S_i | d(x, y) < r_i\}$. Take $r = \min_{\forall i}\{r_i\}$, set $B = \{x \notin S|d(x, y) < r\} \neq \emptyset$. Which means for any $x\notin S$, we can find at least one point $x^\prime \in B$ that for all open set $S_i$, $x^\prime \in S_i$, which makes $x$ a limit point of the complement of the open set. Notice that $x$ is arbitrary, then the collection of $x$, i.e., the complement of $S$ is a closed set.
			\end{proof}

			\begin{remark}
				The empty set is open and closed, the whole space $X$ is open and closed.
			\end{remark}

		\section{Functions, Sequences, Limits and Continuity}

		\section{Differentiation}

		\section{Integration}

		\section{Infinite Series of Constants}

		\section{Power Series}

		\section{Uniform Convergence}

		\section{Arcs and Curves}

		\section{Partial Differentiation}

		\section{Multiple Integrals}

		\section{Improper Integrals}

		\section{Fourier Series}

	\chapter{Review of Probability Theory}
		\section{Relationship between Some Random Variables}
			\begin{figure}[h]
				\centering
				\begin{tikzpicture}[node distance = 1.8cm]
					\node (Pascal) [roundedRectangle] {Pascal($n$, $p$)};
					\node (Poisson) [roundedRectangle, xshift = 6cm] {Poisson($\lambda$)};
					\node (Geometric) [roundedRectangle, below of = Pascal] {Geometric($p$)};
					\node (Binomial) [roundedRectangle, below of = Poisson, xshift = 2cm] {Binomial($n$, $p$)};
					\node (Bernoulli) [roundedRectangle, below of = Poisson, xshift = 7cm] {Bernoulli($p$)};
					\node (Lognormal) [roundedRectangle, below of = Geometric, xshift = 2.5cm] {Lognormal};
					\node (Normal) [roundedRectangle, below of = Geometric, xshift = 7.5cm] {Normal($\mu$, $\sigma^2$)};
					\node (Hypergeometric) [roundedRectangle, below of = Geometric, xshift = 12.5cm] {Hypergeometric($M$, $N$, $K$)};
					\node (SNormal) [roundedRectangle, below of = Lognormal, xshift=2cm] {Normal(0, 1)};
					\node (Beta) [roundedRectangle, below of = Lognormal, xshift = 7cm] {Beta($\alpha$, $\beta$)};
					\node (DExponential) [roundedRectangle, below of = Beta, xshift = 3.8cm] {Double-Exponential($0$, $\lambda$, $\lambda$)};
					\node (Cauchy) [roundedRectangle, below of = SNormal, xshift = -6cm] {Cauchy};
					\node (Gamma) [roundedRectangle, below of = SNormal, xshift = 2cm] {Gamma($r$,$\lambda$)};
					\node (tv) [roundedRectangle, below of = Cauchy, xshift = 1cm] {$t_{(v)}$};
					\node (Chi-squared) [roundedRectangle, below of = Gamma, xshift = -2cm] {Chi-squared($n$)};
					\node (Exponential) [roundedRectangle, below of = Gamma, xshift = 3cm] {Exponential($\lambda$)};
					\node (Fvv) [roundedRectangle, below of = tv, xshift = 1cm] {$F_{(V_{1},V_{2})}$};
					\node (Weibull) [roundedRectangle, below of = Exponential, xshift = -3cm] {Weibull($a$, $b$)};
					\node (SUniform) [roundedRectangle, below of = Exponential, xshift = 3cm] {Uniform($0$, $1$)};
					\node (Uniform) [roundedRectangle, below of = SUniform, xshift = -2cm] {Uniform($a$, $b$)};
					\draw [arrow] (Pascal) to [out = -30, in = 30] node [right] {$n=1$} (Geometric);
					\draw [arrow] (Pascal) -- node [above, align=center, text width=2.5cm] {$\lambda=\frac{n}{p}$ \\ $n\rightarrow \infty$} (Poisson);
					\draw [arrow] (Pascal) to [out = -15, in = 170] node [below, align=center, text width=2.5cm, yshift=0.1cm, xshift=-1.1cm] {$\mu = n(1-p)$ \\ $n\rightarrow \infty$} (Normal);
					\draw [arrow] (Geometric) to [out = 150, in = -150] node [left] {$\sum X_i$} (Pascal);
					\draw [arrow] (Geometric) to [out = -160, in = 180, looseness=6] node [below, yshift = -0.2cm] {$\min(X_i)$} (Geometric);
					\draw [arrow] (Poisson) to [out = 70, in = 110, looseness=4] node [above] {$\sum X_i$} (Poisson);
					\draw [arrow] (Poisson) to [out = -90, in = 145] node [left] {$X|\sum X_i$} (Binomial);
					\draw [arrow] (Poisson) to [out = -150,in = 160] node [left, align=center, yshift=0.6cm] {$\lambda=\sigma^2$ \\ $n\rightarrow \infty$} (Normal);
					\draw [arrow] (Binomial) to [out = 75, in = -40] node [right, align=center] {$\lambda=np$ \\ $n\rightarrow \infty$} (Poisson);
					\draw [arrow] (Binomial) -- node [below] {$n=1$} (Bernoulli);
					\draw [arrow] (Bernoulli) to [out = 165, in = 15] node [above] {$\sum X_i$} (Binomial);
					\draw [arrow] (Binomial) to [out = -70, in = 30, align=center] node [left] {$\sigma^2=np(1-p)$ \\ $\mu=np, n\rightarrow \infty$} (Normal);
					\draw [arrow] (Hypergeometric) to [out = 170, in = -45] node [right] {$p=\frac{M}{N}, n=k, N\rightarrow \infty$} (Binomial);
					\draw [arrow] (Normal) to [out = 0, in = -45, looseness=3] node [right] {$\sum X_i$} (Normal);
					\draw [arrow] (Normal) to [out = 175, in = 5] node [above, xshift=-0.3cm] {$e^{X}$} (Lognormal);
					\draw [arrow] (Normal) to [out = -155, in = 30] node [right, xshift=0.2cm] {$\frac{X-\mu}{\sigma^2}$} (SNormal); 
					\draw [arrow] (Lognormal) to [out = -5, in = -175] node [below] {$\ln X$} (Normal);
					\draw [arrow] (Lognormal) to [out = 160, in = -160, looseness = 3] node [left] {$\prod X_i$} (Lognormal);
					\draw [arrow] (SNormal) to [out = 90, in = -165] node [left, yshift = -0.4cm, xshift = -0.3cm] {$\mu + \sigma X$} (Normal);
					\draw [arrow] (Beta) to [out = 120, in = -90] node [right, yshift=-0.3cm, xshift=0.9cm] {$\alpha, \beta \rightarrow \infty$} (Normal);
					\draw [arrow] (Beta) to [out = 0, in = 30, looseness=2.4] node [right, yshift=1cm, xshift=-3cm] {$\alpha=\beta=1$} (SUniform);
					\draw [arrow] (Gamma) to [out = 100, in = -110] node [right, align = center, yshift=-0.8cm, xshift = -0.1cm] {$\mu=r\lambda$\\$\sigma^2=r\lambda^2$\\$r\rightarrow \infty$} (Normal); 
					\draw [arrow] (Gamma) to [out = 15, in = -90] node [right] {$\frac{X_1}{X_1 + X_2}$} (Beta);
					\draw [arrow] (Gamma) to [out = -165, in = 75] node [right] {$r=\frac{n}{2}, \lambda=2$} (Chi-squared);
					\draw [arrow] (Gamma) to [out = -40, in = 160] node [left] {$r=1$} (Exponential);
					\draw [arrow] (SNormal) to [out = 180, in = 30] node [right] {$\frac{X_1}{X_2}$} (Cauchy);
					\draw [arrow] (SNormal) to [out = -120, in =90] node [left] {$\sum X_i^2$} (Chi-squared);
					\draw [arrow] (Cauchy) to [out = 80, in = 120, looseness = 4] node [above] {$\frac{1}{X}$} (Cauchy);
					\draw [arrow] (Cauchy) to [out = -120, in = -140, looseness = 3] node [below] {$\sum X_i$} (Cauchy);
					\draw [arrow] (tv) to [out = 90, in = -90] node [right, yshift = 0.2cm] {$v=1$} (Cauchy);
					\draw [arrow] (tv) to [out = 60, in = -130] node [left] {$v\rightarrow \infty$} (SNormal);
					\draw [arrow] (tv) to [out = -90, in = 120] node [left] {$X^2$} (Fvv);
					\draw [arrow] (Fvv) to [out = 30, in = -120] node [below] {$V_1X, V_2\rightarrow \infty$} (Chi-squared);
					\draw [arrow] (Chi-squared) to [out = -165, in = 60] node [above] {$\frac{X_1V_2}{X_2V_1}$} (Fvv);
					\draw [arrow] (Chi-squared) to [out = 125, in = 175, looseness = 3] node [above] {$\sum X_i$} (Chi-squared);
					\draw [arrow] (Chi-squared) -- node [below] {$n=2$} (Exponential);
					\draw [arrow] (Exponential) to [out = 140, in = 0] node [right] {$\sum X_i$} (Gamma);
					\draw [arrow] (Exponential) to [out = 50, in = -120] node [below] {$X_1 - X_2$} (DExponential);
					\draw [arrow] (Exponential) to [out = 10, in = -20, looseness = 3] node [right] {$\min X_i$} (Exponential);
					\draw [arrow] (Exponential) to [out = -40, in = 180] node [right] {$e^{-\frac{X}{\lambda}}$} (SUniform);
					\draw [arrow] (Exponential) to [out = -100, in = 0] node[right] {$X^{\frac{1}{b}}$} (Weibull);
					\draw [arrow] (DExponential) to [out = -170, in = 80] node [above] {$|X|$} (Exponential);
					\draw [arrow] (Uniform) to [out = 0, in = -80] node [right] {$a=0, b=1$} (SUniform);
					\draw [arrow] (SUniform) to [out = 50, in = -30] node [below] {$-\lambda \ln X$} (Exponential);
					\draw [arrow] (SUniform) to [out = -110, in = 30] node [left] {$a+(b-a)X$} (Uniform);
					\draw [arrow] (Weibull) to [out = 70, in = -160] node [left] {$b=1$} (Exponential);
				\end{tikzpicture}
				\caption{Relationship between Some Random Variables}
			\end{figure}

		\section{Discrete Random Variables}
			\begin{align}
				\centering
				\begin{tabular}[c]{|c|c|c|c|c|c|}
					\hline
					Distribution & PMF & CDF & Exp. & Var. & MGF \\
					\hline
					Uniform$(a, b)$ & 
					\makecell{$\frac1{b-a+1}$ \\ $x=a, a+1, ..., b$} & 
					\makecell{$\frac{x-a+1}{b-a+1}$ \\ $x=a, a+1, ..., b$} & 
					$\frac{b-a}{2}$ & 
					$\frac{(b-a+1)^2-1}{12}$ & 
					\makecell{$\frac{e^{at}-e^{(b+1)t}}{(b-a+1)(1-e^t)}$ \\ $t \in \mathbb{R}$} \\
					\hline
					Bernoulli$(p)$ & 
					\makecell{$p^x(1-p)^{1-x}$ \\ $x \in \{0,1\}$} &
					$\begin{cases}0, \quad x<0 \\ 
										1-p, \quad 0\le x \le 1 \\
										1, \quad x>1\end{cases}$ & 
					$p$ & 
					$p(1-p)$ & 
					\makecell{$1-p+pe^{t}$ \\ $t\in \mathbb{R}$}\\
					\hline
					Binomial$(n, p)$ &
					\makecell{$\left(\begin{matrix}n\\x\end{matrix}\right)p^x(1-p)^{n-x}$ \\ $x=0,1,...,n$} &
					\makecell{$\sum_{k=0}^{x}\left(\begin{matrix}n\\k\end{matrix}\right)p^k(1-p)^{n-k}$ \\ $x=0,1,...,n$} &
					$np$ &
					$np(1-p)$ &
					\makecell{$(1-p+pe^{t})^n$ \\ $t\in \mathbb{R}$}\\
					\hline
					Poisson$(\mu)$ &
					\makecell{$\frac{\mu^xe^\mu}{x!}$ \\ $x = 0,1,...,n,...$} &
					\makecell{$\frac{\Gamma(x+1, \mu)}{\Gamma(x+1)}$ \\ $x=0,1,...,n,...$} &
					$\mu$ &
					$\mu$ &
					\makecell{$e^{\mu(e^t-1)}$ \\ $t\in \mathbb{R}$} \\
					\hline
					Geometric$(p)$ &
					\makecell{$p(1-p)^x$ \\ $x=0,1,...,n,...$} &
					\makecell{$1-(1-p)^{x+1}$ \\ $x=0,1,...,n,...$} &
					$\frac{1-p}p$ &
					$\frac{1-p}{p^2}$ &
					\makecell{$\frac{p}{1-(1-p)e^t}$ \\ $t < -\ln(1-p)$}\\
					\hline
					Pascal$(n, p)$ &
					\makecell{$\left(\begin{matrix}n-1+x\\x\end{matrix}\right)p^n(1-p)^x$ \\ $x=0,1,2,...,n,...$} &
					\makecell{$1-I_p(k+1,n)$ \\ $x=0,1,2,...,n,...$} &
					$\frac{n(1-p)}{p}$&
					$\frac{n(1-p)}{p^2}$ &
					\makecell{$(\frac{p}{1-(1-p)e^t})^n$ \\ $t < -\ln(1-p)$}\\
					\hline
				\end{tabular}
			\end{align}

		\section{Continuous Random Variables}
			\begin{align}
				\centering
				\begin{tabular}[c]{|c|c|c|c|c|c|}
					\hline
					Distribution & PDF & CDF & Exp. & Var. & MGF \\
					\hline
					Uniform$(a, b)$ & 
					\makecell{$\frac1{b-a}$ \\ $x=[a,b]$} & 
					\makecell{$\frac{x-a}{b-a}$ \\ $x=[a,b]$} & 
					$\frac{b-a}{2}$ & 
					$\frac{(b-a)^2}{12}$ & 
					$\begin{cases}1, \quad t=0 \\ \frac{e^{bt}-e^{at}}{t(b-a)}, \quad t\ne 0\end{cases}$\\
					\hline
					Normal$(\mu, \sigma)$ &
					\makecell{$\frac{1}{\sqrt{2\pi}\sigma}e^{-\frac{(x-\mu)^2}{2\sigma^2}}$ \\ $x\in \mathbb{R}$} &
					\makecell{$\int_{-\infty}^{x}\frac{1}{\sqrt{2\pi}\sigma}e^{-\frac{(x-\mu)^2}{2\sigma^2}}$ \\ $x\in \mathbb{R}$} &
					$\mu$ &
					$\sigma^2$ &
					\makecell{$e^{\frac{t(t\sigma^2+2\mu)}{2}}$ \\ $t\in \mathbb{R}$}\\
					\hline
					Exponential$(\lambda)$ &
					\makecell{$\lambda e^{-\lambda x}$ \\ $x>0$} &
					\makecell{$1-e^{\lambda x}$ \\ $x>0$} &
					$\frac{1}{\lambda}$ &
					$\frac{1}{\lambda^2}$ &
					\makecell{$\frac{1}{1-\frac{t}{\lambda}}$ \\ $t<\lambda$}\\
					\hline
					Erlang$(n, \lambda)$ &
					\makecell{$\frac{\lambda^n x^{n-1} e^{-\lambda x}}{(n-1)!}$ \\ $x>0$} &
					\makecell{$1-\sum_{i=0}^{n-1}\frac{\lambda^n x^n e^{-\lambda x}}{n!}$ \\ $x>0$} &
					$\frac{n}{\lambda}$ &
					$\frac{n}{\lambda^2}$ &
					\makecell{$\frac{1}{(1-\frac{t}{\lambda})^n}$ \\ $t<\lambda$} \\
					\hline
				\end{tabular}
			\end{align}

	\chapter{Review of Statistics}
		\section{Classical Statistics}

		\section{Bayesian Statistics}

\end{document}