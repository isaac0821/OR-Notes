\part{Preliminary Topics}
	\chapter{Introduction to Optimization}
		\section{Optimization Model}
			The following is the basic forms of terminology:
				\begin{align}
					\text{(P)} \quad \text{min} \quad & f(x)  \\
								\text{s.t.} \quad & g_i(x)\le 0, \quad i=1,2,...,m \\
											& h_j(x)=0, \quad j=1,2,...,l \\
											& x \in X 
				\end{align}
				We have
				\begin{itemize}
					\item - $x\in R^n \rightarrow X \subseteq R^m$
					\item $g_i(x)$ are called inequality constraints
					\item $h_j(x)$ are called equality constraints
					\item $X$ is the domain of the variables (e.g. cone, polygon, $\{0, 1\}^n$, etc.)
					\item Let $F$ be the feasible region of $(P)$:
					\begin{itemize}
						\item $x^0$ is a feasible solution iff $x^0\in F$
						\item $x^*$ is an optimized solution iff $x^* \in F$ and $f(x^*)\le f(x^0), \forall x^0 \in F$ (for minimized problem)
					\end{itemize}
				\end{itemize}

			\notice{Not every $(P)$ has a feasible region, we can have $F=\emptyset$. Even if $F \ne \emptyset$, there might not be an solution to $P$, e.g. unbounded. If $(P)$ has optimized solution(s), it could be 1) Unique 2) Infinite number of solution 3) Finite number of solution}

			Types of Optimization Problem
			\begin{itemize}
				\item $m = l = 0$, $x \in R^n$, unconstrained problem
				\item $m + l > 0$, constrained problem
				\item $f(x), g_i(x), h_j(x)$ are linear, Linear Optimization
				\begin{itemize}
					\item If $X=R^n$, Linear Programming
					\item If $X$ is discrete,  Discrete Optimization
					\item If $X \subseteq Z^n$, Integer Programming
					\item If $X\in \{0,1\}^n$, Binary Programming
					\item If $X\in Z^n \times R^m$, Mixed Integer Programming
				\end{itemize}
			\end{itemize}

	\chapter{Review of Linear Algebra}
		\section{Field}
			\begin{definition}[Field]
				Let $F$ denote either the set  of real numbers or the set of complex numbers.
				\begin{itemize}
					\item Addition is commutative: $x + y = y + x, \forall x, y \in F$
					\item Addition is associative: $x + (y + z) = (x + y) + z, \forall x, y, z \in F$
					\item Element 0 exists and unique: $\exists 0, x + 0 = x, \forall x \in F$
					\item To each $x \in F$ there corresponds a unique element $(-x) \in F$ such that $x + (-x) = 0$
					\item Multiplication is commutative: $xy = yx, \forall x, y \in F$
					\item Multiplication is associative: $x(yz) = (xy)z, \forall x, y, z \in F$
					\item Element 1 exists and unique: $\exists 1, x1=x, \forall x \in F$
					\item To each $x\neq 0 \in F$ there corresponds a unique element $x^{-1} \in F$ that $xx^{-1} = 1$
					\item Multiplication distributes over addition: $x(y + z) = xy + xz, \forall x, y, z \in F$
				\end{itemize}
				Suppose one has a set $F$ of objects $x, y, z, ...$ and two operations on the elements of $F$ as follows. The first operation, called addition, associates with each pair of elements $x, y \in F$ an element $(x + y)\in F$; the second operation, called multiplication, associates with each pair $x, y$ an element $xy \in F$; and these two operations satisfy all conditions above. The set $F$, together with these two operations, is then called a \textbf{field}.
			\end{definition}

			\begin{definition}[Subfield]
				A \textbf{subfield} of the field $C$ is a set $F$ of complex numbers which itself is a field.
			\end{definition}

			\begin{example}
				The set of integers is not a field.
			\end{example}

			\begin{example}
				The set of rational numbers is a field.
			\end{example}

			\begin{example}
				The set of all complex numbers of the form $x + y\sqrt{2}$ where $x$ and $y$ are rational, is a subfield of $\mathbb{C}$.
			\end{example}

			\notice{In this note, we (...Lan) assume that the field involved is a subfield of the complex numbers $\mathbb{C}$. More generally, if $F$ is a field, it may be possible to add the unit $1$ to itself a finite number of times and obtain $0$, which does not happen in the subfield of $\mathbb{C}$. If it does happen in $F$, the least $n$ such that the sum of $n$ 1's is 0 is called \textbf{characteristic} of the field $F$. If it does not happen, then $F$ is called a field of \textbf{characteristic zero}.}

		\section{Real Vector Spaces}

		\section{Linear, Conic, Affine, and Convex Combinations}

		\section{Determinants}

		\section{Inner Products}
			\begin{definition}[Inner Product]
				Let $F$ be the field of real numbers or the field of complex numbers, and $V$ a vector space over $F$. An \textbf{inner product} on $V$ is a function which assigns to each ordered pair of vectors $\alpha$, $\beta$ in $V$ a scalar $<\alpha|\beta>$ in $F$ in such a way that $\forall \alpha, \beta, \gamma \in V, c \in \mathbb{R}$ that
				\begin{itemize}
					\item $<\alpha+\beta|\gamma> = <\alpha|\gamma> + <\beta|\gamma>$
					\item $<c\alpha|\beta> = c<\alpha|\beta>$
					\item $<\alpha|\beta> = \overline{<\beta|\alpha>}$
					\item $<\alpha|\alpha> \ge 0$, $<\alpha|\alpha> = 0$ iff $\alpha = \mathbf{0}$
				\end{itemize}
				Furthermore, the above properties imply that
				\begin{itemize}
					\item $<\alpha|c\beta+\gamma> = \bar{c}<\alpha|\beta> + <\alpha|\gamma>$
				\end{itemize}
			\end{definition}

			\begin{definition}
				On $F^n$ there is an inner product which we call the \textbf{standard inner product}. It is defined on $\mathbf{\alpha} = (x_1, x_2, ..., x_n)$ and $\mathbf{\beta} = (y_1, y_2, ..., y_n)$ by
				\begin{equation}
					<\alpha|\beta> = \sum_j x_j \bar{y_j}
				\end{equation}
				For $F = \mathbb{R}^n$
				\begin{equation}
					<\alpha|\beta> = \sum_j x_j y_j
				\end{equation}
				In the real case, the standard inner product is often called the dot product and denoted by $\alpha \cdot \beta$
			\end{definition}

			\begin{example}
				For $\mathbf{\alpha} = (x_1, x_2)$ and $\mathbf{\beta} = (y_1, y_2)$ in $\mathbb{R}^2$, the following is an inner product.
				\begin{equation}
					<\alpha|\beta> = x_1y_1 - x_2y_1 - x_1y_2 + 4x_2y_2
				\end{equation}
			\end{example}

			\begin{example}
				For $\mathbb{C}^{n\times n}$, 
				\begin{equation}
					<\mathbf{A}|\mathbf{B}> = trace(\mathbf{B}^* \mathbf{A})
				\end{equation}
				is an inner product, where
				\begin{equation}
					\mathbf{A}^*_{ij} = \bar{\mathbf{A}}_{ji} \quad (\textbf{conjugate transpose})
				\end{equation}
				For $\mathbb{R}^{n\times n}$,
				\begin{equation}
					<\mathbf{A}|\mathbf{B}> = trace(\mathbf{B}^T \mathbf{A}) = \sum_j (AB^T)_{jj} = \sum_j\sum_k A_{jk}B_{jk}
				\end{equation}
			\end{example}
			
		\section{Norms}

		\section{Eigenvectors and Eigenvalues}
			\begin{definition}
				If $\mathbf{A}$ is an $n \times n$ matrix, then a nonzero vector $\mathbf{x} \in \mathbb{R}^n$ is called an \textbf{eigenvector} of $\mathbf{A}$ if $\mathbf{Ax}$ is a scalar multiple of $\mathbf{x}$, i.e.
				\begin{equation}
					\mathbf{Ax} = \lambda \mathbf{x}
				\end{equation}
				for some scalar $\lambda$. The scalar $\lambda$ is called \textbf{eigenvalue} of $\mathbf{A}$ and the vector $\mathbf{x}$ is said to be an \textbf{eigenvector corresponding to $\lambda$}
			\end{definition}

			\begin{theorem}[Characteristic Equation]
				If $\mathbf{A}$ is an $n \times n$ matrix, then $\lambda$ is an eigenvalue of $\mathbf{A}$ iff
				\begin{equation}
					\det(\lambda I - A) = 0
				\end{equation}
			\end{theorem}

			\begin{corollary}
				\begin{equation}
					\sum \lambda_A = tr(\mathbf{A})
				\end{equation}
			\end{corollary}

			\begin{corollary}
				\begin{equation}
					\prod \lambda_A = \det(\mathbf{A})
				\end{equation}
			\end{corollary}

			\notice{Gaussian elimination changes the eigenvalues.}


		\section{Decompositions}

	\chapter{Review of Real Analysis}
		\section{Sequences and Series}

	\chapter{Review of Topology}
		\section{Open Sets and Closed Sets}
			\begin{definition}[Metric Space]
				A \textbf{metric space} is a set $X$ where we have a notion of distance. That is, if $x, y \in X$, then $d(x, y)$ is the distance between $x$ and $y$. The particular distance function must satisfy the following conditions:
				\begin{itemize}
					\item $d(x, y) > 0, \forall x, y \in X$
					\item $d(x, y) = 0 \iff x=y$
					\item $d(x, y) = d(y, x)$
					\item $d(x, z) \le d(x, y) + d(y, z)$
				\end{itemize}
			\end{definition}

			\begin{definition}[Ball]
				Let $X$ be a metric space. A \textbf{ball} $B$ of radius $r$ around a point $x \in X$ is
				\begin{equation}
					B = \{y \in X|d(x, y) < r\}
				\end{equation}
			\end{definition}

			\begin{definition}[Open Set]
				A subset $O \subseteq X$ is \textbf{open} if $\forall x \in O, \exists r, B=\{x\in X|d(x, y) < r\} \subseteq O$
			\end{definition}

			\begin{theorem}
				The union of any collection if open sets is open.
			\end{theorem}

			\begin{theorem}
				The intersection of any finite number of open sets is open.
			\end{theorem}

			\begin{remark}
				The intersection of infinite number of open sets is not necessarily open.
			\end{remark}

			\begin{definition}[Limit point]
				A point $z$ is a \textbf{limit point} for a set $A$ if every open set $U$ that $z\in U$ intersects $A$ in a point other than $z$.
			\end{definition}

			\notice{$z$ is not necessarily in $A$.}

			\begin{definition}[Closed Set]
				A set $C$ is \textbf{closed} iff it contains all of its limit points.
			\end{definition}

			\begin{theorem}
				Every intersection of closed sets is closed.
			\end{theorem}

			\begin{theorem}
				Every finite union of closed sets is closed.
			\end{theorem}

			\begin{remark}
				The union of infinite number of closed sets is not necessarily closed.
			\end{remark}

			\begin{theorem}
				A set $C$ is a closed set if $X \setminus C$ is open
			\end{theorem}

			\begin{remark}
				The empty set is open and closed, the whole space $X$ is open and closed.
			\end{remark}