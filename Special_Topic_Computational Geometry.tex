\documentclass[10pt, onecolumn]{book}
\author{Lan Peng, PhD Student\\ \\Department of Industrial and Systems Engineering\\University at Buffalo, SUNY\\lanpeng@buffalo.edu}
\title{Notes for Operations Research \& More}

\usepackage{amsmath}
\usepackage{amssymb}
\usepackage{amsfonts}
\usepackage{graphicx}
\usepackage{amsthm}
\usepackage{color}
\usepackage{tabularx}
\usepackage{diagbox}
\usepackage{bm}
\usepackage{mathrsfs}
\usepackage{hyperref}
\usepackage{longtable}
\usepackage{makecell}
\usepackage{lscape}

\usepackage{algorithm}
\usepackage{algpseudocode}
\algtext*{EndWhile}
\algtext*{EndIf}
\algtext*{EndFor}

\usepackage[
	letterpaper,
	left=2cm,
	right=2cm,
	top=2cm,
	bottom=2cm]{geometry}
	\setlength{\parindent}{0pt}

\usepackage{subcaption}
\usepackage{tikz}
	\usetikzlibrary{chains, arrows,shapes,matrix}
	\usetikzlibrary{decorations.pathmorphing} 
	\usepgflibrary{plotmarks}
	\usetikzlibrary{patterns}  
	\usetikzlibrary{positioning} 
	\tikzstyle{roundedRectangle} = [
		rectangle, 
		rounded corners, 
		minimum width=3cm, 
		minimum height=1cm, 
		text centered, 
		draw=black
	]
	\tikzstyle{io} = [
		trapezium, 
		trapezium left angle=70, 
		trapezium right angle=110, 
		minimum width=3cm, 
		minimum height=1cm, 
		text centered, 
		draw=black
	]
	\tikzstyle{process} = [
		rectangle, 
		minimum width=2cm, 
		minimum height=1cm, 
		text centered, 
		draw=black, 
		inner sep=0.1cm
	]
	\tikzstyle{decision} = [
		diamond, 
		minimum width=2cm, 
		minimum height=0cm, 
		text centered, 
		draw=black, 
		inner sep=0cm
	]
	\tikzstyle{arrow} = [
		thick,
		->,
		>=stealth
	]
	\tikzstyle{link} = [
		thick, 
		-
	]
	\tikzstyle{matchedLink} = [
		decorate, 
		decoration={snake}
	]
	\tikzstyle{circleNode} = [
		circle, 
		minimum size = 0.7cm, 
		text centered, 
		draw=black, 
		inner sep=0.1cm
	]
	\tikzstyle{solidNode} = [
		circle, 
		minimum size = 0.1cm, 
		fill=black
	]
	\tikzstyle{smallSolidNode} = [
		circle, 
		minimum size = 0.03cm, 
		fill=black
	]
	\tikzstyle{rectangleCell} = [
		rectangle, 
		minimum width=0.8cm, 
		text centered, 
		draw=black
	]
	\tikzstyle{rowArray} = [
		matrix of nodes, 
		nodes = {draw}, 
		row 1/.style = {
			nodes = {
				draw = none
			}
		}
	]
	\tikzstyle{colArray} = [
		matrix of nodes, 
		nodes = {draw}, 
		column 2/.style = {
			nodes = {
				draw = none
			}
		}
	]


\theoremstyle{definition}
	\newtheorem{definition}{Definition}[section]
	\newtheorem*{example}{Example}
	\newtheorem{problem}{Problem}[chapter]
	\newtheorem*{solution}{Solution}
	\newtheorem{hypothesis}{Hypothesis}[section]
\theoremstyle{plain}
	\newtheorem{theorem}{Theorem}[chapter]
	\newtheorem{corollary}{Corollary}[theorem]
	\newtheorem{lemma}[theorem]{Lemma}
	\newtheorem{conjecture}{Conjecture}
	\newtheorem{proposition}{Proposition}
\theoremstyle{remark}
	\newtheorem*{remark}{Remark}

\usepackage[square,numbers]{natbib}
	\bibliographystyle{plainnat}
	\bibpunct[, ]{(}{)}{,}{a}{}{,}
	\def\bibfont{\small}
	\def\bibsep{\smallskipamount}
	\def\bibhang{24pt}
	\def\newblock{\ }
	\def\BIBand{and}

\newcommand{\todo}[1]{
	\vspace{5 mm}
	\par
	\noindent
	\marginpar{\textsc{to do}}
	\framebox{
		\begin{minipage}[c]{0.95 \textwidth}
		\tt
		\begin{center} 
			#1
		\end{center}
		\end{minipage}
	}
	\vspace{5 mm}
	\par
}

\newcommand{\notice}[1]{
	\vspace{2 mm}
	\par
	\noindent
	\colorbox{gray!15}{
		\centering
		\begin{minipage}[c]{1 \textwidth}
			\textbf{Notice:}~#1
		\end{minipage}
	}
	\vspace{2 mm}
	\par
}

\newcommand{\fixme}[1]{
	{\color{red}#1}
	\marginpar{
		\textsc{
			\color{red}
			FIXME
		}
	}
}
\begin{document}
\part*{Special Topic: Computational Geometry}
	\addcontentsline{toc}{part}{Special Topic: Computational Geometry}
	\chapter{Convex Hull}
		\section{Computing Slope Statistics}

		\section{Convexity}

		\section{Graham's Scan}

		\section{Turning and orientations}
	\chapter{Intersections}

	\chapter{Triangulation and Partitioning}
		\section{Polygon Triangulation}
			\subsection{Types of Polygons}
				\begin{definition}[simple polygon]
					A \textbf{simple polygon} is a closed polygonal curve without self-intersection.
				\end{definition}

				\begin{figure}[h!]
					\centering
					\begin{tikzpicture}[scale=0.6]
						\draw (0, 0) -- (3, -1) -- (4, 3) -- (2, 4) -- (0, 0);
						\draw (6, 3) -- (8, -1) -- (9, 2) -- (5.5, 0) -- (6, 3);
						\node at (2, -1.5) [below] {Simple Polygon};
						\node at (7.5, -1.5) [below] {Non-simple Polygon};
					\end{tikzpicture}
				\end{figure}

				Polygons are basic building blocks in most geometric applications. It can model arbitrarily complex shapes, and apply simple algorithms and algebraic representation/manipulation.

			\subsection{Triangulation}
				\begin{definition}[Triangulation]
					\textbf{Triangulation} is to partition polygon $P$ into non-overlapping triangles using diagonals only. It reduces complex shapes to collection of simpler shapes. Every simple $n$-gon admits a triangulation which has $n-2$ triangles.				
				\end{definition}

				\begin{figure}[h!]
					\centering
					\begin{tikzpicture}[scale=0.6]
						\draw [thick] (0, 0) -- (3, 4) -- (2, 5) -- (3.5, 5.5) -- (1, 6.4) -- (-1, 5) -- (-2, 4.5) -- (-1.3, 2.4) -- (-2, 1) -- (0, 0);
						\draw (-2, 1) -- (3, 4);
						\draw (-1.3, 2.4) -- (3, 4);
						\draw (-2, 4.5) -- (3, 4);
						\draw (-1, 5) -- (3, 4);
						\draw (-1, 5) -- (2, 5);
						\draw (2, 5) -- (1, 6.4);
						\node at (0, 0) [below] {Triangulation};
					\end{tikzpicture}
				\end{figure}

				\begin{theorem}
					Every polygon has a triangulation				
				\end{theorem}

				\begin{lemma}
					Every polygon with more than three vertices has a diagonal.
				\end{lemma}

				\begin{proof}
					(by Meisters, 1975) Let $P$ be a polygon with more than three vertices. Every vertex of a $P$ is either \textit{convex} or \textit{concave}. W.L.O.G.(any polygon must has convex corner) Assume $p$ is a convex vertex. Denote the neighbors of $p$ as $q$ and $r$. If $\bar{qr}$ is a diagonal, done, and we call $\triangle{pqr}$ is an \textit{ear}. If $\triangle{pqr}$ is not an ear, it means at least one vertex is inside $\triangle{pqr}$, assume among those vertexes inside $\triangle{pqr}$, $s$ is a vertex closest to $p$, then $\bar{ps}$ is a diagonal.
				\end{proof}

			\subsection{Art Gallery Theorem}
				\begin{theorem}
					Every $n$-gon can be guarded with $\lfloor \frac{n}{3} \rfloor$ vertex guards
				\end{theorem}

				\begin{lemma}
					Triangulation graph can be 3-colored.
				\end{lemma}

				\begin{problem}
					The floor plan of an art gallery modeled as a simple polygon with $n$ vertices, there are guards which is stationed at fixed positions with 360 degree vision but cannot see through the walls. How many guards does the art gallery need for the security? (Fun fact: This problem was posted to Vasek Chvatal by Victor Klee in 1973).				
				\end{problem}

				\begin{proof}
					- $P$ plus triangulation is a planar graph\\
					- 3-coloring means there exist a 3-partition for vertices that no edge or diagonal has both endpoints within the same set of vertices.\\
					- Proof by Induction:\\
					\indent - Remove an ear (there will always exist ear) \\
					\indent - Inductively 3-color the rest\\
					\indent - Put ear back, coloring new vertex with the label not used by the boundary diagonal.
				\end{proof}

			\subsection{Triangulation Algorithms}

	\chapter{Voronoi Diagrams}

	\chapter{Arrangement and Duality}

	\chapter{Delaunay Triangulations}

	\chapter{Search}

	\chapter{Motion Planning}

	\chapter{Quadtrees}

	\chapter{Visibility Graphs}

\end{document}