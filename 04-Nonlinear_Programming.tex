\part{Nonlinear Programming}
	\chapter{Introduction and Movtivation}
		\section{Basic concept}
			\begin{align}
				\min \quad &f(x) \\
				\text{s.t.} \quad & g_i(x) \le 0, \quad \forall i = 1, ..., m\\
								  & h_i(x) = 0, \quad \forall i = 1, ..., m\\
								  &x \in X
			\end{align}

			Iso-value curve\\
			Given a function $f(x): R^n \rightarrow R$, the iso-value curve of level k, denoted by $L_k$ is the set of points for which $f(x)=k$, Formally,
			\begin{equation}
				L_k = \{x \in R^n | f(x)=k\}
			\end{equation}

			Observations:\\
			\indent - An optimization problem may not have a feasible solution\\
			\indent - Even if the problem has a feasible solution, the problem may not have an optimal solution\\
			\indent - If an optimal solution exists, then\\
			\indent \indent - if may be unique
			\indent \indent - it may be a convex set
			\indent \indent - it may be serveral sets

			The following two problems are equivalent
			\begin{align}
				\min \quad & x_1 + x_2\\
				\text{s.t.} \quad & x_1 + 2x_2\le 3 \\
				                  & x_1, x_2 \in \{0, 1\}
			\end{align}
			and 
			\begin{align}
				\min \quad & x_1 + x_2\\
				\text{s.t.} \quad & x_1 + 2x_2\le 3 \\
				                  & x_1 \ge 0 \\
				                  & x_2 \ge 0 \\
				                  & x_1(1 - x_1) = 0\\
				                  & x_2(1 - x_2) = 0
			\end{align}		

			\section{Example - Braess Paradox}

			- Situation 1, minimizing total time
			\begin{align}
				\min \quad & \frac{x_1t_1 + x_2t_2 + x_3t_3}{x_1 + x_2 + x_3}\\
				\text{s.t.} \quad & x_1 + x_2 + x_3 = 1000\\
				                  & t_1 = 1 + (x_1 + x_2) / 1000 + 2\\
				                  & t_2 = 1 + (x_1 + x_2) / 1000 + 0.25 + (x_2 + x_3)/1000 + 1\\
				                  & t_3 = 2 + (x_2 + x_3) / 1000 + 1
			\end{align}

			- Situation 2, selfish for driver
			\begin{align}
				\text{s.t.} \quad & x_1 + x_2 + x_3 = 1000\\
				                  & t_1 = 1 + (x_1 + x_2) / 1000 + 2\\
				                  & t_2 = 1 + (x_1 + x_2) / 1000 + 0.25 + (x_2 + x_3)/1000 + 1\\
				                  & t_3 = 2 + (x_2 + x_3) / 1000 + 1\\
				                  & x_1 t_1 \le x_1 t_2 \\
				                  & x_1 t_1 \le x_1 t_3 \\
				                  & x_2 t_2 \le x_2 t_1 \\
				                  & x_2 t_2 \le x_2 t_3 \\
				                  & x_3 t_3 \le x_3 t_1 \\
				                  & x_3 t_3 \le x_3 t_2
			\end{align}

			- Situation 3, remove one of the arc, remove $x_2$ and all related constraints

		\section{Newsvendor problem}
			Parameter
			\begin{itemize}
				\item for each newspaper, cost is \$c
				\item vendor determine the number of newspaper to buy as $N$
				\item each news paper selled for \$p
				\item each news paper has a salvage value of \$v
				\item for demend scenario $d_i, i\in N$, the probability is $\pi_i \in N$
			\end{itemize}

			Decision Vars
			\begin{itemize}
				\item number of newspaper to buy, $q$
				\item number of newspaper returned, $r_i, i \in N$
				\item number of newspaper sold, $s_i, i\in N$
			\end{itemize}

			Constraints
			\begin{itemize}
				\item $q = s_i + r_i$
				\item $s_i = \min \{d_i, q\}$
				\item $q \in Z_+ \cup \{0\}$
				\item $r_i, s_i \in Z_+ \cup \{0\}$
			\end{itemize}

			Objective Fuction
			\begin{equation}
				\max \quad \sum_{i \in N} \pi_i (ps_i + vr_i) - cq
			\end{equation}

		\section{Portfolio optimization}
			Parameters
			\begin{itemize}
				\item A, set of assets
				\item $\bar{r}$, desired portfolio's return
				\item $r_i$, expected return of $i\in A$
				\item $\sigma_{ij}^2$, covariance between $i$ and $j$, $i, j \in A$
			\end{itemize}

			Decision Vars
			\begin{itemize}
				\item $x_i$ \% of total budget to invest to asset $i \in A$
			\end{itemize}

			Constraints
			\begin{itemize}
				\item $\sum_{i \in A} x_i = 1$
				\item $\sum_{i \in A} \alpha_i x_i \ge \bar{r}$
				\item Nonnegativity
			\end{itemize}

			Objective function
			\begin{equation}
				\min \quad \sum_{i \in A}\sum_{j \in A} \sigma_{ij}^2 x_i x_j
			\end{equation}

		\section{curve fitting}
			
	\chapter{Background}
		\section{Operations}
			A vector addition $\oplus$ has such feature
			\begin{itemize}
				\item $\oplus$ is closed
				\item $\oplus$ is commutative
				\item $\oplus$ is associative
				\item $\oplus$ has an identity element
				\item $\oplus$ has an inverse
			\end{itemize}

			A scalar multiplication $\otimes$ has the following features
			\begin{itemize}
				\item $\otimes$ is closed
				\item $\otimes$ is distributive wrt $\oplus$
				\item $\otimes$ is distributive wrt scalar addition, i.e. $(\lambda + \mu) \otimes v = (\lambda \otimes v) \oplus (\mu \otimes v$
				\item $\otimes$ is distributive with multiplication
			\end{itemize}

			\begin{equation}
				u \oplus v = \left[\begin{align}u_1 \\ u_1\end{align}\right] \oplus \left[\begin{align}v_1 \\ v_1\end{align}\right]
			\end{equation}

			For $W_1 = \{w \in R^3 | 3w_1 + 2w_2 + w_3 = 0\}$
			Check if this is closed to $\oplus$

			\begin{align}
				a, b &\in W_1 \\
				3a_1 + 2a_2 + a_3 & = 0\\
				3b_1 + 2b_2 + b_3 & = 0\\
				c & = a + b
				c & = 3 (a_1 + b_1) + ... = 0
			\end{align}

	\chapter{KKT Optimality Conditions}

	\chapter{Lagrangian Duality}

	\chapter{Unconstrained Optimization}

	\chapter{Penalty and Barrier Functions}