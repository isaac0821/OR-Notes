\documentclass[10pt, onecolumn]{book}
\author{Lan Peng, PhD Student\\ \\Department of Industrial and Systems Engineering\\University at Buffalo, SUNY\\lanpeng@buffalo.edu}
\title{Notes for Operations Research \& More}

\usepackage{amsmath}
\usepackage{amssymb}
\usepackage{amsfonts}
\usepackage{graphicx}
\usepackage{amsthm}
\usepackage{color}
\usepackage{tabularx}
\usepackage{diagbox}
\usepackage{bm}
\usepackage{mathrsfs}
\usepackage{hyperref}
\usepackage{longtable}
\usepackage{makecell}
\usepackage{lscape}

\usepackage{algorithm}
\usepackage{algpseudocode}
\algtext*{EndWhile}
\algtext*{EndIf}
\algtext*{EndFor}

\usepackage[
	letterpaper,
	left=2cm,
	right=2cm,
	top=2cm,
	bottom=2cm]{geometry}
	\setlength{\parindent}{0pt}

\usepackage{subcaption}
\usepackage{tikz}
	\usetikzlibrary{chains, arrows,shapes,matrix}
	\usetikzlibrary{decorations.pathmorphing} 
	\usepgflibrary{plotmarks}
	\usetikzlibrary{patterns}  
	\usetikzlibrary{positioning} 
	\tikzstyle{roundedRectangle} = [
		rectangle, 
		rounded corners, 
		minimum width=3cm, 
		minimum height=1cm, 
		text centered, 
		draw=black
	]
	\tikzstyle{io} = [
		trapezium, 
		trapezium left angle=70, 
		trapezium right angle=110, 
		minimum width=3cm, 
		minimum height=1cm, 
		text centered, 
		draw=black
	]
	\tikzstyle{process} = [
		rectangle, 
		minimum width=2cm, 
		minimum height=1cm, 
		text centered, 
		draw=black, 
		inner sep=0.1cm
	]
	\tikzstyle{decision} = [
		diamond, 
		minimum width=2cm, 
		minimum height=0cm, 
		text centered, 
		draw=black, 
		inner sep=0cm
	]
	\tikzstyle{arrow} = [
		thick,
		->,
		>=stealth
	]
	\tikzstyle{link} = [
		thick, 
		-
	]
	\tikzstyle{matchedLink} = [
		decorate, 
		decoration={snake}
	]
	\tikzstyle{circleNode} = [
		circle, 
		minimum size = 0.7cm, 
		text centered, 
		draw=black, 
		inner sep=0.1cm
	]
	\tikzstyle{solidNode} = [
		circle, 
		minimum size = 0.1cm, 
		fill=black
	]
	\tikzstyle{smallSolidNode} = [
		circle, 
		minimum size = 0.03cm, 
		fill=black
	]
	\tikzstyle{rectangleCell} = [
		rectangle, 
		minimum width=0.8cm, 
		text centered, 
		draw=black
	]
	\tikzstyle{rowArray} = [
		matrix of nodes, 
		nodes = {draw}, 
		row 1/.style = {
			nodes = {
				draw = none
			}
		}
	]
	\tikzstyle{colArray} = [
		matrix of nodes, 
		nodes = {draw}, 
		column 2/.style = {
			nodes = {
				draw = none
			}
		}
	]


\theoremstyle{definition}
	\newtheorem{definition}{Definition}[section]
	\newtheorem*{example}{Example}
	\newtheorem{problem}{Problem}[chapter]
	\newtheorem*{solution}{Solution}
	\newtheorem{hypothesis}{Hypothesis}[section]
\theoremstyle{plain}
	\newtheorem{theorem}{Theorem}[chapter]
	\newtheorem{corollary}{Corollary}[theorem]
	\newtheorem{lemma}[theorem]{Lemma}
	\newtheorem{conjecture}{Conjecture}
	\newtheorem{proposition}{Proposition}
\theoremstyle{remark}
	\newtheorem*{remark}{Remark}

\usepackage[square,numbers]{natbib}
	\bibliographystyle{plainnat}
	\bibpunct[, ]{(}{)}{,}{a}{}{,}
	\def\bibfont{\small}
	\def\bibsep{\smallskipamount}
	\def\bibhang{24pt}
	\def\newblock{\ }
	\def\BIBand{and}

\newcommand{\todo}[1]{
	\vspace{5 mm}
	\par
	\noindent
	\marginpar{\textsc{to do}}
	\framebox{
		\begin{minipage}[c]{0.95 \textwidth}
		\tt
		\begin{center} 
			#1
		\end{center}
		\end{minipage}
	}
	\vspace{5 mm}
	\par
}

\newcommand{\notice}[1]{
	\vspace{2 mm}
	\par
	\noindent
	\colorbox{gray!15}{
		\centering
		\begin{minipage}[c]{1 \textwidth}
			\textbf{Notice:}~#1
		\end{minipage}
	}
	\vspace{2 mm}
	\par
}

\newcommand{\fixme}[1]{
	{\color{red}#1}
	\marginpar{
		\textsc{
			\color{red}
			FIXME
		}
	}
}
\title{Notes for Classic IP \& CO Paper List}
\begin{document}
	\maketitle
	\today

	\clearpage
	\thispagestyle{plain}
	\par\vspace*{.35\textheight}{\centering \textit{To My Beloved Motherland China}\par}

	\tableofcontents

	% Levels of Contents 
	% \part*{[A paper]}
	% 	\chapter{[Main section]}
	% 		\section{[Main theorem and proof]}

	\chapter[Partitioning Procedures for Solving Mixed-Variables Programming Problems]
		{Partitioning Procedures for Solving Mixed-Variables Programming Problems \\[\bigskipamount]
		\Large J. F. Benders \\[\bigskipamount]
		\large \textit{Numerische Mathematik}, 1962}\label{chp:Benders-Decom}

		\section{Introduction \& Preliminaries}
			The mixed-variables programming problems considered is the following
			\begin{align*}
				(P) \quad \max \quad & \mathbf{c}^\top \mathbf{x} + f(\mathbf{y})\\
				\text{s.t.} \quad & \mathbf{Ax} + F(\mathbf{y}) \le \mathbf{b}\\
				& \mathbf{x} \in \mathbb{R}^p \\
				& \mathbf{y} \in S
			\end{align*}

			in which
			\begin{itemize}
				\item $S$ - is an arbitrary subset of $\mathbb{R}^q$
				\item $\mathbf{A}$ - is an $m \times p$ matrix
				\item $f(\mathbf{y})$ - is a scalar function defined on $S$
				\item $F(\mathbf{y})$ - is an $m$-component vector function defined on $S$
				\item $\mathbf{b}$ - RHS, $\mathbf{b} \in \mathbb{R}^m$
				\item $\mathbf{c}$ - Cost coefficient, $\mathbf{c} \in \mathbb{R}^p$
			\end{itemize}

			\begin{example}
				MIP, LP
			\end{example}

			The basic idea is to partition a given problem into two sub problems, a programming problem ($F(\mathbf{y})$) and a LP ($\mathbf{Ax}$). This happens when the problem indicates a natural partition of the variables.

			Denote $\mathbf{u}, \mathbf{v}, \mathbf{z}$ as vectors in $\mathbb{R}^m$, and $u_0$, $x_0$, and $z_0$ as scalars.

			For $\mathbf{A}$ and $\mathbf{c}$ defined previously, we define
			\begin{itemize}
				\item the convex polyhedral cone $C$ in $\mathbb{R}^{m + 1}$ by
				\begin{equation}
					C = \{\begin{pmatrix}u_0 \\ \mathbf{u}\end{pmatrix} | \mathbf{A}^\top \mathbf{u} - \mathbf{c} u_0 \ge \mathbf{0}, \ \mathbf{u} \ge \mathbf{0}, \ u_0 \ge 0\}
				\end{equation}
				\item the convex polyhedral cone $C_0$ in $\mathbb{R}^m$ by
				\begin{equation}
					C_0 = \{\mathbf{u} | \mathbf{A}^\top \mathbf{u} \ge \mathbf{0}, \ \mathbf{u} \ge \mathbf{0}\}
				\end{equation}
				\item the convex polyhedral $P$ in $\mathbb{R}^m$ by
				\begin{equation}
					P = \{\mathbf{u} | \mathbf{A}^\top \mathbf{u} \ge \mathbf{c}, \ \mathbf{u} \ge \mathbf{0}\}
				\end{equation}
			\end{itemize}

		\section{A partitioning theorem}
			Rewrite (P) into an equivalent form as following
			\begin{align*}
				(P^\prime) \quad \max \quad & x_0\\
				\text{s.t.} \quad & x_0 - \mathbf{c}^\top \mathbf{x} - f(\mathbf{y}) \le 0\\
				& \mathbf{Ax} + F(\mathbf{y}) \le \mathbf{b}\\
				& \mathbf{x} \ge \mathbf{0} \\
				& \mathbf{y} \in S
			\end{align*}

			Clearly, $(x_0^*, \mathbf{x}^*, \mathbf{y}^*)$ is optimal for $(P^\prime)$ iff $x_0^* = \mathbf{c}^\top \mathbf{x}^* + f(\mathbf{y}^*)$ and $(\mathbf{x}^*, \mathbf{y}^*)$ is optimal for $(P)$.

			For one of the subproblem of $(P)$, the LP, we have
			\begin{align*}
				(subLP) \quad \max \quad & \mathbf{c}^\top \mathbf{x}\\
				\text{s.t.} \quad & \mathbf{Ax} \le \mathbf{b^\prime}\\
								  & \mathbf{x} \ge \mathbf{0} \\
								  & \mathbf{x} \in \mathbb{R}^p
			\end{align*}
			take the dual of $(subLP)$, we have
			\begin{align*}
				(dualLP) \quad \min \quad & \mathbf{u^\top b^\prime}\\
				\text{s.t.} \quad & \mathbf{u}^\top \mathbf{A} \ge \mathbf{c}\\
								  & \mathbf{u} \ge \mathbf{0}\\
								  & \mathbf{u} \in \mathbb{R}^m
			\end{align*}

	\chapter[The Traveling-Salesman Problem and Minimum Spanning Tree]
		{The Traveling-Salesman Problem and Minimum Spanning Tree \\[\bigskipamount]
		\Large Michael Held and Richard M. Karp \\[\bigskipamount]
		\large \textit{Operations Research}, 1969}\label{chp:TSP-LR}

		\section{The Traveling-Salesman Problem and a Related Spanning-Tree Problem}
			\begin{definition}[1-tree]
				In graph $G = (V, E)$, where $V = \{1, 2, \cdots, n\}$, a 1-tree consists of a tree on the vertex set $\{2, 3, \cdots, n\}$, together with two distinct edges at vertex 1.
			\end{definition}

			Thus, a 1-tree has a single cycle, this cycle contains vertex 1 and vertex 1 always has degree 2. A minimal weighted 1-tree can be found by constructing a minimum spanning tree on the vertex set $\{2, 3, \cdots, n\}$, and then adjoining two edges of lowest weight at vertex 1.

			Also notice that every tour is a 1-tree, and a 1-tree is a tour iff each of its vertices has degree 2. If a minimum-weight 1-tree is a tour, it is the solution of the TSP.

			\begin{example} An example of 1-tree can be found in figure \ref{HKBound:1_tree}, solid arcs are minimum spanning tree of $\{2, 3, \cdots, n\}$ and two dashed arcs links the MST to vertex 1 with minimal cost.
				\begin{figure}[!ht]
					\centering
					\begin{tikzpicture}[node distance=1.5cm]
						\node [circleNode] (3) {3};
						\node [circleNode, left of=3] (6) {6};
						\node [circleNode, below of=3, xshift=-1cm] (4) {4};
						\node [circleNode, below of=3, xshift=2cm] (2) {2};
						\node [circleNode, below of=4] (1) {1};
						\node [circleNode, right of=1] (5) {5};
						\node [circleNode, right of=2] (7) {7};
						\draw [link] (3) -- (6);
						\draw [link] (3) -- (4);
						\draw [link] (3) -- (2);
						\draw [link] (2) -- (7);
						\draw [link] (2) -- (5);
						\draw [link, dashed] (1) -- (4);
						\draw [link, dashed] (1) -- (5);
					\end{tikzpicture}
					\caption{1-tree}
					\label{HKBound:1_tree}
				\end{figure}
			\end{example}

			\begin{lemma}
				Let $\mathbf{\pi} = (\pi_1, \pi_2, \cdots, \pi_n)$ be a real $n$-vector. If $C^*$ is a minimum-weight tour with respect to the edge weights $c_{ij}$, then it is also a minimum-weight tour $C^\prime$ with respect to the edge weight $c_{ij} + \pi_i + \pi_j$.
			\end{lemma}

			\begin{proof}
				For tour $C$, the weight is $C = \sum_{(i, j) \in C} c_{ij}$. Therefore $C^\prime - C^* = 2 \sum_{i = 1}^n \pi_i$, which is a constant.
			\end{proof}

			Change the costs from $c_{ij}$ to $c_{ij} + \pi_i + \pi_j$ only changes its minimum spanning 1-tree. Introduce a gap function $f(\pi)$, which is the cost of a minimum-weight tour minus the cost of a minimum-weighted 1-tree both with respect to the weights $c_{ij} + \pi_i + \pi_j$. Notice that if $f(\pi) = 0$ then we found optimal tour for TSP, thus, we consider the problem of finding $\min_\pi f(\pi)$, where
			\begin{align*}
				f(\pi) &= W + 2 \sum_{i = 1}^n \pi_i - \min_k (c_k + \sum_{i = 1}^n \pi_i d_{ik})\\
					   &= W - \min_k [c_k + \sum_{i = 1}^n (d_{ik} - 2)]
			\end{align*}
			in which
			\begin{itemize}
				\item $W$ - is the weight of a minimum tour with respect to the weights $c_{ij}$
				\item $c_k$ - is the weight of the $k$th 1-tree with respect to the weight $c_{ij}$. Notice that the 1-trees of the graph are indexed by $1, 2, \dots, q$ with $k$ as a generic index.
				\item $d_{ik}$ - is the degree of vertex $i$ in the $k$th 1-tree.
			\end{itemize}

			The goal is to minimize $f(\pi)$ over $\pi$, which is equivalent to
			\begin{align*}
				\max \quad & w\\
				\text{s.t.} \quad & w \le c_k + \sum_{i = 1}^n (d_{ik} - 2) \quad \forall k
			\end{align*}
			Dualizing, we obtain
			\begin{align*}
				\min \quad & \sum_{k} c_k y_k\\
				\text{s.t.} \quad & \sum_k y_k = 1\\
				& y_k \ge 0 \quad \forall k\\
				& \sum_k (2 - d_{ik}) y_k = 0 \quad i = 2, 3, \ldots, n-1
			\end{align*}

			Notice that this LP model seeks a minimum-weight ``convex combination of 1-trees'' such that each vertex has, on average, degree two.

		\section{Relation to a Linear Program}
			The following is DFJ formulation for TSP
			\begin{align*}
				\min \quad & \sum_{1 \le i < j \le n}c_{ij} x_{ij}\\
				\text{s.t.} \quad & \sum_{i < j} x_{ij} + \sum_{i > j} x_{ji} = 2, \quad i \in \{1, 2, \cdots, n\}\\
								  & \sum_{\substack{i \in S \\ j \notin S \\ i < j}} x_{ij} \le |S| - 1, \quad S \subset \{2, 3, \cdots, n\}\\
								  & x_{ij} \in \{0, 1\}
			\end{align*}

			Equivalently, we can transform it into
			\begin{align}
				\min \quad & \sum_{1 \le i < j \le n}c_{ij} x_{ij}\\
				\text{s.t.} \quad & \sum_{i < j} x_{ij} + \sum_{i > j} x_{ji} = 2, \quad i \in \{2, 3, \cdots, n - 1\} \label{con:TSP-LR:v2toNminus1}\\				
								  & \sum_j x_{1j} = 2 \label{con:TSP-LR:v1}\\
								  & \sum_{1 \le i < j \le n} x_{ij} = n \label{con:TSP-LR:sumAllEdge}\\
								  & \sum_{\substack{i \in S \\ j \notin S \\ i < j}} x_{ij} \le |S| - 1, \quad S \subset \{2, 3, \cdots, n\}\label{con:TSP-LR:subtour}\\
								  & x_{ij} \le 1 \label{con:TSP-LR:upperX}\\
								  & x_{ij} \ge 0\\
								  & x_{ij} \in \{0, 1\}
			\end{align}

			Denote constraints (\ref{con:TSP-LR:v2toNminus1}) by $\mathbf{Ax} = \mathbf{b}$, and the constraints (\ref{con:TSP-LR:v1}), (\ref{con:TSP-LR:sumAllEdge}), (\ref{con:TSP-LR:subtour}) and (\ref{con:TSP-LR:upperX}) by $\mathbf{A}^\prime \mathbf{x} = \mathbf{b}^\prime$.

		\section{A Column-Generation Technique}

		\section{An Ascent Method}

		\section{A Branch-and-Bound Method}

		\section{Minimal 1-Trees and Matroids}

		\section{Other Applications}
\end{document}