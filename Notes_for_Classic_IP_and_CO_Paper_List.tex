\documentclass[10pt, onecolumn]{book}
\author{Lan Peng, PhD Student\\ \\Department of Industrial and Systems Engineering\\University at Buffalo, SUNY\\lanpeng@buffalo.edu}
\title{Notes for Operations Research \& More}

\usepackage{amsmath}
\usepackage{amssymb}
\usepackage{amsfonts}
\usepackage{graphicx}
\usepackage{amsthm}
\usepackage{color}
\usepackage{tabularx}
\usepackage{diagbox}
\usepackage{bm}
\usepackage{mathrsfs}
\usepackage{hyperref}
\usepackage{longtable}
\usepackage{makecell}
\usepackage{lscape}

\usepackage{algorithm}
\usepackage{algpseudocode}
\algtext*{EndWhile}
\algtext*{EndIf}
\algtext*{EndFor}

\usepackage[
	letterpaper,
	left=2cm,
	right=2cm,
	top=2cm,
	bottom=2cm]{geometry}
	\setlength{\parindent}{0pt}

\usepackage{subcaption}
\usepackage{tikz}
	\usetikzlibrary{chains, arrows,shapes,matrix}
	\usetikzlibrary{decorations.pathmorphing} 
	\usepgflibrary{plotmarks}
	\usetikzlibrary{patterns}  
	\usetikzlibrary{positioning} 
	\tikzstyle{roundedRectangle} = [
		rectangle, 
		rounded corners, 
		minimum width=3cm, 
		minimum height=1cm, 
		text centered, 
		draw=black
	]
	\tikzstyle{io} = [
		trapezium, 
		trapezium left angle=70, 
		trapezium right angle=110, 
		minimum width=3cm, 
		minimum height=1cm, 
		text centered, 
		draw=black
	]
	\tikzstyle{process} = [
		rectangle, 
		minimum width=2cm, 
		minimum height=1cm, 
		text centered, 
		draw=black, 
		inner sep=0.1cm
	]
	\tikzstyle{decision} = [
		diamond, 
		minimum width=2cm, 
		minimum height=0cm, 
		text centered, 
		draw=black, 
		inner sep=0cm
	]
	\tikzstyle{arrow} = [
		thick,
		->,
		>=stealth
	]
	\tikzstyle{link} = [
		thick, 
		-
	]
	\tikzstyle{matchedLink} = [
		decorate, 
		decoration={snake}
	]
	\tikzstyle{circleNode} = [
		circle, 
		minimum size = 0.7cm, 
		text centered, 
		draw=black, 
		inner sep=0.1cm
	]
	\tikzstyle{solidNode} = [
		circle, 
		minimum size = 0.1cm, 
		fill=black
	]
	\tikzstyle{smallSolidNode} = [
		circle, 
		minimum size = 0.03cm, 
		fill=black
	]
	\tikzstyle{rectangleCell} = [
		rectangle, 
		minimum width=0.8cm, 
		text centered, 
		draw=black
	]
	\tikzstyle{rowArray} = [
		matrix of nodes, 
		nodes = {draw}, 
		row 1/.style = {
			nodes = {
				draw = none
			}
		}
	]
	\tikzstyle{colArray} = [
		matrix of nodes, 
		nodes = {draw}, 
		column 2/.style = {
			nodes = {
				draw = none
			}
		}
	]

	\theoremstyle{definition}
		\newtheorem{definition}{Definition}[section]
		\newtheorem*{example}{Example}
		\newtheorem{problem}{Problem}[chapter]
		\newtheorem*{solution}{Solution}
	\theoremstyle{plain}
		\newtheorem{theorem}{Theorem}[chapter]
		\newtheorem{corollary}{Corollary}[theorem]
		\newtheorem{lemma}[theorem]{Lemma}
		\newtheorem{conjecture}{Conjecture}
		\newtheorem{proposition}{Proposition}
	\theoremstyle{remark}
		\newtheorem*{remark}{Remark}

\newcommand{\todo}[1]{
	\vspace{5 mm}
	\par
	\noindent
	\marginpar{\textsc{to do}}
	\framebox{
		\begin{minipage}[c]{0.95 \textwidth}
		\tt
		\begin{center} 
			#1
		\end{center}
		\end{minipage}
	}
	\vspace{5 mm}
	\par
}

\newcommand{\notice}[1]{
	\vspace{2 mm}
	\par
	\noindent
	\colorbox{gray!15}{
		\centering
		\begin{minipage}[c]{1 \textwidth}
			\textbf{Notice:}~#1
		\end{minipage}
	}
	\vspace{2 mm}
	\par
}

\newcommand{\alert}[1]{
	{\color{red}#1}
}

\newcommand{\edited}[1]{
	{\color{blue}#1}
}

\newcommand{\fixme}[1]{
	{\color{red}#1}
	\marginpar{
		\textsc{
			\color{red}
			fixme
		}
	}
}
\title{Notes for Classic IP \& CO Paper List}
\begin{document}
	\maketitle
	\today

	\clearpage
	\thispagestyle{plain}
	\par\vspace*{.35\textheight}{\centering \textit{To My Beloved Motherland China}\par}

	\tableofcontents

	% Levels of Contents 
	% \part*{[A paper]}
	% 	\chapter{[Main section]}
	% 		\section{[Main theorem and proof]}

	\chapter[The Traveling-Salesman Problem and Minimum Spanning Tree]
		{The Traveling-Salesman Problem and Minimum Spanning Tree \\[\bigskipamount]
		\Large Michael Held and Richard M. Karp \\[\bigskipamount]
		\large \textit{Operations Research}, 1969}

		\section{The Traveling-Salesman Problem and a Related Spanning-Tree Problem}
			\begin{definition}[1-tree]
				In graph $G = (V, E)$, where $V = \{1, 2, \cdots, n\}$, a 1-tree consists of a tree on the vertex set $\{2, 3, \cdots, n\}$, together with two distinct edges at vertex 1.
			\end{definition}

			Thus, a 1-tree has a single cycle, this cycle contains vertex 1 and vertex 1 always has degree 2. A minimal weighted 1-tree can be found by constructing a minimum spanning tree on the vertex set $\{2, 3, \cdots, n\}$, and then adjoining two edges of lowest weight at vertex 1.

			Also notice that every tour is a 1-tree, and a 1-tree is a tour iff each of its vertices has degree 2. If a minimum-weight 1-tree is a tour, it is the solution of the TSP.

			\begin{example} An example of 1-tree can be found in figure \ref{HKBound:1_tree}, solid arcs are minimum spanning tree of $\{2, 3, \cdots, n\}$ and two dashed arcs links the MST to vertex 1 with minimal cost.
				\begin{figure}[!ht]
					\centering
					\begin{tikzpicture}[node distance=1.5cm]
						\node [circleNode] (3) {3};
						\node [circleNode, left of=3] (6) {6};
						\node [circleNode, below of=3, xshift=-1cm] (4) {4};
						\node [circleNode, below of=3, xshift=2cm] (2) {2};
						\node [circleNode, below of=4] (1) {1};
						\node [circleNode, right of=1] (5) {5};
						\node [circleNode, right of=2] (7) {7};
						\draw [link] (3) -- (6);
						\draw [link] (3) -- (4);
						\draw [link] (3) -- (2);
						\draw [link] (2) -- (7);
						\draw [link] (2) -- (5);
						\draw [link, dashed] (1) -- (4);
						\draw [link, dashed] (1) -- (5);
					\end{tikzpicture}
					\caption{1-tree}
					\label{HKBound:1_tree}
				\end{figure}
			\end{example}

			\begin{lemma}
				Let $\mathbf{\pi} = (\pi_1, \pi_2, \cdots, \pi_n)$ be a real $n$-vector. If $C^*$ is a minimum-weight tour with respect to the edge weights $c_{ij}$, then it is also a minimum-weight tour $C^\prime$ with respect to the edge weight $c_{ij} + \pi_i + \pi_j$.
			\end{lemma}

			\begin{proof}
				For tour $C$, the weight is $C = \sum_{(i, j) \in C} c_{ij}$. Therefore $C^\prime - C^* = 2 \sum_{i = 1}^n \pi_i$, which is a constant.
			\end{proof}

			Change the costs from $c_{ij}$ to $c_{ij} + \pi_i + \pi_j$ only changes its minimum spanning 1-tree. Introduce a gap function $f(\pi)$, which is the cost of a minimum-weight tour minus the cost of a minimum-weighted 1-tree both with respect to the weights $c_{ij} + \pi_i + \pi_j$. Notice that if $f(\pi) = 0$ then we found optimal tour for TSP, thus, we consider the problem of finding $\min_\pi f(\pi)$, where
			\begin{equation}
				f(\pi) = 
			\end{equation}
\end{document}