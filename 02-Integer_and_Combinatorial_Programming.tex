\documentclass[10pt, onecolumn]{book}
\author{Lan Peng, PhD Student\\ \\Department of Industrial and Systems Engineering\\University at Buffalo, SUNY\\lanpeng@buffalo.edu}
\title{Notes for Operations Research \& More}

\usepackage{amsmath}
\usepackage{amssymb}
\usepackage{amsfonts}
\usepackage{graphicx}
\usepackage{amsthm}
\usepackage{color}
\usepackage{tabularx}
\usepackage{diagbox}
\usepackage{bm}
\usepackage{mathrsfs}
\usepackage{hyperref}
\usepackage{longtable}
\usepackage{makecell}
\usepackage{lscape}

\usepackage{algorithm}
\usepackage{algpseudocode}
\algtext*{EndWhile}
\algtext*{EndIf}
\algtext*{EndFor}

\usepackage[
	letterpaper,
	left=2cm,
	right=2cm,
	top=2cm,
	bottom=2cm]{geometry}
	\setlength{\parindent}{0pt}

\usepackage{subcaption}
\usepackage{tikz}
	\usetikzlibrary{chains, arrows,shapes,matrix}
	\usetikzlibrary{decorations.pathmorphing} 
	\usepgflibrary{plotmarks}
	\usetikzlibrary{patterns}  
	\usetikzlibrary{positioning} 
	\tikzstyle{roundedRectangle} = [
		rectangle, 
		rounded corners, 
		minimum width=3cm, 
		minimum height=1cm, 
		text centered, 
		draw=black
	]
	\tikzstyle{io} = [
		trapezium, 
		trapezium left angle=70, 
		trapezium right angle=110, 
		minimum width=3cm, 
		minimum height=1cm, 
		text centered, 
		draw=black
	]
	\tikzstyle{process} = [
		rectangle, 
		minimum width=2cm, 
		minimum height=1cm, 
		text centered, 
		draw=black, 
		inner sep=0.1cm
	]
	\tikzstyle{decision} = [
		diamond, 
		minimum width=2cm, 
		minimum height=0cm, 
		text centered, 
		draw=black, 
		inner sep=0cm
	]
	\tikzstyle{arrow} = [
		thick,
		->,
		>=stealth
	]
	\tikzstyle{link} = [
		thick, 
		-
	]
	\tikzstyle{matchedLink} = [
		decorate, 
		decoration={snake}
	]
	\tikzstyle{circleNode} = [
		circle, 
		minimum size = 0.7cm, 
		text centered, 
		draw=black, 
		inner sep=0.1cm
	]
	\tikzstyle{solidNode} = [
		circle, 
		minimum size = 0.1cm, 
		fill=black
	]
	\tikzstyle{smallSolidNode} = [
		circle, 
		minimum size = 0.03cm, 
		fill=black
	]
	\tikzstyle{rectangleCell} = [
		rectangle, 
		minimum width=0.8cm, 
		text centered, 
		draw=black
	]
	\tikzstyle{rowArray} = [
		matrix of nodes, 
		nodes = {draw}, 
		row 1/.style = {
			nodes = {
				draw = none
			}
		}
	]
	\tikzstyle{colArray} = [
		matrix of nodes, 
		nodes = {draw}, 
		column 2/.style = {
			nodes = {
				draw = none
			}
		}
	]


\theoremstyle{definition}
	\newtheorem{definition}{Definition}[section]
	\newtheorem*{example}{Example}
	\newtheorem{problem}{Problem}[chapter]
	\newtheorem*{solution}{Solution}
	\newtheorem{hypothesis}{Hypothesis}[section]
\theoremstyle{plain}
	\newtheorem{theorem}{Theorem}[chapter]
	\newtheorem{corollary}{Corollary}[theorem]
	\newtheorem{lemma}[theorem]{Lemma}
	\newtheorem{conjecture}{Conjecture}
	\newtheorem{proposition}{Proposition}
\theoremstyle{remark}
	\newtheorem*{remark}{Remark}

\usepackage[square,numbers]{natbib}
	\bibliographystyle{plainnat}
	\bibpunct[, ]{(}{)}{,}{a}{}{,}
	\def\bibfont{\small}
	\def\bibsep{\smallskipamount}
	\def\bibhang{24pt}
	\def\newblock{\ }
	\def\BIBand{and}

\newcommand{\todo}[1]{
	\vspace{5 mm}
	\par
	\noindent
	\marginpar{\textsc{to do}}
	\framebox{
		\begin{minipage}[c]{0.95 \textwidth}
		\tt
		\begin{center} 
			#1
		\end{center}
		\end{minipage}
	}
	\vspace{5 mm}
	\par
}

\newcommand{\notice}[1]{
	\vspace{2 mm}
	\par
	\noindent
	\colorbox{gray!15}{
		\centering
		\begin{minipage}[c]{1 \textwidth}
			\textbf{Notice:}~#1
		\end{minipage}
	}
	\vspace{2 mm}
	\par
}

\newcommand{\fixme}[1]{
	{\color{red}#1}
	\marginpar{
		\textsc{
			\color{red}
			FIXME
		}
	}
}
\begin{document}
\part{Integer and Combinatorial Programming}
	\chapter{Polyhedral Analysis}
		\section{Polyhedral and Dimension}
			\subsection{Polyhedral, Hyperplanes and Half-spaces}
				- A \textbf{polyhedron} is a set of the form $\{x\in \mathbb{R}^n|Ax\le b\}=\{x \in \mathbb{R}^n | a^ix\le b^i, \forall i \in M\}$, where $A \in \mathbb{R}^{m\times n}$ and $b \in \mathbb{R}^m$\\
				- A polyhedron $P \subset \mathbb{R}^n$ is \textbf{bounded} if there exists a constant $K$ such that $|x_i|<K, \forall x \in P, \forall i \in [1, n]$, in this case the polyhedron is call \textbf{polytopes}\\
				- The lower-bound of $K$ is called \textbf{diagonal} denoted by $d$\\

			\subsection{Open, Close Sets: boundary and interior}
				- Denote $N_\epsilon = \{y\in \mathbb{R}^n|\lVert y-x\rVert < \epsilon \}$ as the \textbf{neighborhood} of $x\in \mathbb{R}^n$\\
				- Given $S\subseteq \mathbb{R}^n$, x belongs to the \textbf{interior} of $S$, denoted by $int(S)$ if there is $\epsilon > 0$ such that $N_\epsilon(x) \le S$\\
				- $S$ is said to be an \textbf{open set} iff $S=int(S)$\\
				- $x$ belongs to the \textbf{boundary} $\partial S$ if $\forall \epsilon >0$, $N_\epsilon(x)$ contains at least one point in $S$ and a point not in $S$\\
				- $x\in S$ belongs to the \textbf{closure} of $S$, denoted $cl(s)$ if $\forall \epsilon > 0$, $N_\epsilon(x) \cap S = \emptyset$
				- $S$ is called \textbf{closed} iff $S=cl(S)$\\
				- In IP, LP, MIP, etc. we always work with close set. No \lq\lq{}$<$\rq\rq{} or \lq\lq{}$>$\rq\rq{}

			\subsection{Hyperplane and half-space}
				- A \textbf{hyperplane} is $\{x\in \mathbb{R}^n|a^Tx=b\}$\\
				- A \textbf{half-space} is  $\{x\in \mathbb{R}^n|a^Tx\le b\}$

			\subsection{Dimension of Polyhedral}
				- A polyhedron $P$ is \textbf{dimension} $k$, denoted $dim(P)=k$, if the maximum number of affinely independent points in $P$ is $k+1$\\
				- A polyhedron $P\subseteq \mathbb{R}^n$ is \textbf{full-dimensional} if $dim(P) = n$\\
				- Let:\\
				\indent - $M=\{1, 2, ..., m\}$\\
				\indent - $M^= = \{i \in M | a_ix=b_i, \forall x \in P\}$, i.e. the equality set\\
				\indent - $M^\le = M \setminus  M^=$, i.e. the inequality set\\
				- Let $(A^=, b^=)$, $(A^\le, b^\le)$ be the corresponding rows of $(A, b)$\\
				- If $P\subseteq \mathbb{R}^n$, then $dim(P) = n - rank(A^=, b^=)$\\
				- To proof a constraint $(A^=, b^=)$ is an equality constraint, we need to proof all point in the closure of $P$ satisfied the constraint, to proof it is not an equality constraint, we need to find one point that is not in the hyperplane.

			\subsection{Dimension and Rank}
				- $x\in P$ is called an \textbf{inner point} of $P$ if $a^ix < b_i, \forall i \in M^\le$\\
				- $x\in P$ is called an \textbf{interior point} of $P$ if $a^ix<b_i, \forall i \in M$\\
				- Every nonempty polyhedron has at least one inner point\\
				- A polyhedron has an interior point iff $P$ is full-dimensional, i.e., there is no equality constraint

		\section{Face and Facet}
			\subsection{Valid Inequalities and Faces}
				- The inequality denoted by $(\pi, \pi_o)$ is called a \textbf{valid inequality} for $P$ if $\pi x \le \pi_0, \forall x \in P$\\
				- Note that $(\pi, \pi_0)$ is a valid inequality iff $P$ lies in the half-space $\{x\in \mathbb{R}^n|Ax\le b\}$\\
				\begin{figure}[!ht]
				\centering
				\begin{tikzpicture}[scale=0.4]
					\draw (0,0) -- (3, -0.4) -- (3.3, 4.1) -- (-0.2, 4.1) -- (0, 0);
					\draw (-2, 2) -- (2, -2);
					\draw (4.9, 4.1) -- (-1.8, 4.1);
					\draw (4, 1) -- (2, 6);
					\draw [arrow] (-1.8, 4.1) -> (-1.8, 3.3);
					\draw [arrow] (4.9, 4.1) -> (4.9, 3.3);
					\draw [arrow] (-2, 2) -> (-1, 3);
					\draw [arrow] (2, -2) -> (3, -1);
					\draw [arrow] (4, 1) -> (3, 0.6);
					\draw [arrow] (2, 6) -> (1, 5.6);
					\node at (1, -1) [left] {valid inequality};
					\node at (4.9, 4) [right] {valid inequality (induce a facet)};
					\node at (2, 6) [right] {invalid inequality};
				\end{tikzpicture}
				\caption{Example of valid/invalid inequality}
				\end{figure}
				- If $(\pi, \pi_0)$ is a valid inequality for $P$ and $F=\{x\in P|\pi x=x_0\}$, $F$ is called a \textbf{facet} of $P$ and we say that $(\pi, \pi_0)$ \textbf{represents} or \textbf{defines} $F$\\
				- A face is said to be \textbf{proper} if $F\ne \emptyset$ and $F\ne P$\\
				- The face represented by $(\pi, \pi_0)$ is nonempty iff $\max \{\pi x |x\in P\}=\pi_0\}$\\
				- If the face $F$ is nonempty, we say it \textbf{supports} $P$\\
				- Let $P$ be a polyhedron with equality set $M^=$. If 
				\begin{equation}F=\{x\in P | \pi^T x = \pi_0\}  \end{equation}
				is not empty, then $F$ is a polyhedron. Let 
				\begin{equation}M^= \subseteq M_F^=, M_F^{\le}=M \setminus M_F^= \end{equation}
				then 
				\begin{equation}F=\{x | a_i^T x=b_i, \forall i \in M_F^=, a_i^T x \le b_i, \forall i \in M_f^{\le}\} \end{equation}

			\subsection{Facet}
				- A face $F$ is said to be a \textbf{facet} of $P$ if $dim(F) = dim(P)-1$\\
				- Facets are all we need to describe polyhedral\\
				- If $F$ is a facet of $P$, then in any description of $P$, there exists some inequality representing $F$\\
				- Every inequality that represents a face that is not a facet is unnecessary in the description of $P$
				- Every full-dimensional polyhedron $P$ has a unique (up to scalar multiplication) representation that consists of one inequality representing each facet of $P$\\
				- If $dim(P) = n-k$ with $k>0$, then $P$ is described by a maximal set of linearly independent rows of $(A^=, b^=)$, as well as one inequality representing each facet of $P$

			\subsection{Proving Facet}
				To prove an inequality $\sum_i a_i x_i \le b_i$ is facet inducing for a $D$ dimensional polyhedral, we need to prove there are $D$ affinely independent vectors in $\sum_i a_i x_i = b_i$

			\subsection{Domination}
				$\Pi x\le \Pi_0$ dominates $Mx\le M_0$ if
				\begin{equation}
					\begin{cases}
						\Pi \ge \mu M, \mu > 0\\
						\Pi_0 \le \mu M_0, \mu > 0\\
						(\Pi, \Pi_0) \ne (M, M_0)
					\end{cases}
				\end{equation}

	\chapter{Branch and Bound}
		\section{LP based Branch and Bound}
			\subsection{Idea of Divide and Conquer}
				For each iteration, divide the feasible region of LP into two feasible parts and an infeasible part, solve the LP in those parts.\\
				\begin{figure}[!ht]
					\centering
					\begin{tikzpicture}[scale=0.5]
						\draw [<->] (10, 0) -- (0,0) -- (0,10);
						\draw (0, 8.5) -- (10, 2.5);
						\draw (2, 8) -- (9, 0);
						\draw [dashed] (3,0) -- (3,10);
						\draw [dashed] (4,0) -- (4,10);
						\node at (1.5, 5) {$S_1$};
						\node at (3.5, 4) {$S_2$};
						\node at (5, 3) {$S_3$}; 
					\end{tikzpicture}
					\caption{Divide and Conquer}
				\end{figure}
				In this iteration, the original feasible region have been partition into three parts, where $S_2$ is infeasible for IP because there is not integer point in it. We continue the iteration for $S_1$ and $S_2$. Each partition is suppose to give a new upper bound / lower bound and reduce the infeasible space.\\
				If the temp optimal integer in $S_1$ is larger than the LP relaxation in $S_3$, we can cut $S_3$.\\
				For each iteration, we use dual simple method, for the following two reasons:\\
				\indent - We can process new constraint very fast\\
				\indent - Always gives us a valid bound.\\

			\subsection{Relation Between LP Relaxation and IP}
				Let
				\begin{align}
					Z_{IP} &=\max_{x\in S} cx, \quad \text{where } s \text{ is a set of integer solutions} \\
					Z_{LP} &=\max cx, \quad \text{the LP relaxation of IP}  
				\end{align}
				then
				\begin{align}
					Z_{IP} &= \max_{1\le i \le k} \{\max_{x \in S_i} cx \} \\
					\text{iff} \quad S&=\bigcup_{1\le i \le k} S_i 
				\end{align}
				Notice that $S_i$ don\rq{}t need to be disjointed.\\
				\textbf{Important!} (For maximization problem)\\
				\indent - Any feasible solution provides a lower bound $L$, which is also the \textit{Prime Bound}\\
				\begin{equation}\hat{x}\in S \rightarrow Z_{IP}\ge c\hat{x}\end{equation}
				\indent - After branching, solving the LP relaxation over sub-feasible-region $S_i$ produces an upper bound, which is also the \textit{Dual Bound}, on each sub-problem\\
				\indent - If $u(S_i)\le L$, remove $S_i$\\
				\indent - LP can produce the first upper bound, but there might be possible to find other upper bound with other method (e.g. Lagrangian relaxation)

			\subsection{LP feasibility and IP(or MIP) feasibility}
				Solve the LP relaxation, one of the following things can happen\\
				\indent - LP is infeasible $\rightarrow$ MIP is infeasible\\
				\indent - LP is unbounded $\rightarrow$ MIP is infeasible or unbounded\\
				\indent - LP has optimal solution $\hat{x}$ and $\hat{x}$ are integer ($\hat{x} \in S$), $\rightarrow$ $Z_{IP} = Z_{LP}$\\
				\indent - LP has optimal solution $\hat{x}$ and $\hat{x}$ are not integer ($\hat{x} \notin S$), now defines a new upper bound, $Z_{LP} \ge Z_{IP}$\\
				If the first three happens, stop, if the fourth happens, we branch and recursively solve the sub-problems.

		\section{Terminology in Branch and Bound}
			- If we picture the sub-problems, they will form a \textbf{search tree} (typically a binary tree)\\
			- Each node in the search tree is a \textbf{sub-problem}\\
			- Eliminating a node is called \textbf{pruning}, we also stop considering its children\\
			- A sub-problem that has not being processed is called a \textbf{candidate}, we keep a list of candidates

		\section{Bounding}
		 	\textbf{Notice!} this section is for maximization, if it is for minimization, reverse upper bound and lower bound.
			\subsection{Upper Bound}
				- Upper bound it the Prime bound. which means it has to be a feasible solution\\
				- Some methods to get an upper bound:\\
				\indent - Rounding\\
				\indent - Heuristic\\
				\indent - Meta-heuristic

			\subsection{Lower Bound}
				- Lower Bound is the Dual bound,we can use LP relaxation to get it\\
				- The tighter the better, LP is better\\

		\section{Branch and Bound Algorithm}
			\begin{algorithm}[!ht]
				\caption{Branch and Bound}
				\begin{algorithmic}[1]
					\State find a feasible solution as the initial Lower bound $L$
					\State put the original LP relaxation in candidate list $S$
					\While {$S \ne \emptyset$}
						\State select a problem $\hat{S}$ from $S$
						\State solve the LP relaxation of $\hat{S}$ to obtain $u(\hat{S})$
						\If {LP is infeasible}
							\State $\hat{S}$ pruned by infeasibility
						\ElsIf {LP is unbounded}
							\State $\hat{S}$ pruned by unboundness or infeasibility
						\ElsIf{LP $u(\hat{S}) \le L$}
							\State $\hat{S}$ pruned by bound
						\ElsIf{LP $u(\hat{S}) > L$}
							\If {$\hat{x}\in S$}
								\State $u(\hat{S})$ becomes new $L$, $L=u(\hat{S})$
							\ElsIf {$\hat{x}\notin S$}
								\State branch and add the new sub-problems to $S$
								\If {LP $u(\hat{S})$ is at current best upper bound}
									\State set $U=u(\hat{S})$
								\EndIf
							\EndIf
						\EndIf
					\EndWhile
					\If {Lower bound exists}
						\State find the optimal at $L$
					\Else
						\State Infeasible
					\EndIf
				\end{algorithmic}
			\end{algorithm}

		\section{The goal of Branching}
			- Divide the problem into easier sub-problems\\
			- We want to chose the branching variables that minimize the sum of the solution times of the sub-problems\\
			- If after branching the $u(S_i)$ changes a lot,\\
			\indent - I can find a good L first\\
			\indent - The branch may get worse than the current bound first\\
			- Instead of solving the potential two branches for all candidates to optimality, solve a few iterations of the dual simplex, each iteration of pivoting yields an upper bound.

		\section{Choose Branching Variables}
			\subsection{The Most Violated Integrality constraint}
				Pick the $j$ of which $x_j - \lfloor \hat{x_j} \rfloor$ is closer to 0.5

			\subsection{Strong Branching}
				Select a few candidates $(K)$, create all sub-problems for each of these variables, run a few dual simplex iterations to see the improved bounds, select the variable with the best bounds.\\
				for variable $x_j\in K$, we branch and do a few iterations to find two reductions of gaps, i.e. $D_j^+$ and $D_j^-$,
				\begin{figure}[!ht]
					\centering
					\begin{tikzpicture}[scale=0.2]
						\draw (0, 5) circle [radius=0.5];
						\draw (-5, 0) circle [radius=0.5];
						\draw (5, 0) circle [radius=0.5];
						\draw (0.353, 4.647) -- (4.647, 0.353);
						\draw (-0.353, 4.647) -- (-4.647, 0.353);
						\node [above] at (0, 5.5) {$\hat{x_j}$};
						\node [left] at (-2.5, 2.5) {$x_j \le \lfloor \hat{x_j} \rfloor$};
						\node [right] at (2.5, 2.5) {$x_j \ge \lceil \hat{x_j} \rceil$};
						\draw [->] (-5, -1) -- (-5, -3);
						\draw [->] (5, -1) -- (5, -3);
						\node [below] at (-5, -3) {$D_j^+$};
						\node [below] at (5, -3) {$D_j^-$};
					\end{tikzpicture}
					\caption{Strong Branching}
				\end{figure}

			\subsection{pseudo-cost Branching}
				Pseudo-cost is an estimate of per-unit change in the objective function, for each variable
				\begin{equation}\begin{cases}P_j^+, & \text{bound reduction if rounded up} \\ P_j^-, & \text{bound reduction if rounded down}\end{cases}\end{equation}
				define $f_j = x_j -\lfloor x_j \rfloor$
				\begin{equation}\begin{cases}D_j^+ = P_j^+ (1-f_j) \\ D_j^- = P_j^- f_j\end{cases}\end{equation}

		\section{Choose the Node to Branch}
			\subsection{Update After Branching}
				For those variables in $K$ find the \\
				- $\max \{\min\{ {D_j^+},  {D_j^-}\}\}$, or\\
				- $\max \{\max\{ {D_j^+},  {D_j^-}\}\}$, or\\
				- $\max \{\frac{D_j^+ + D_j^-}{2}\}$, or\\
				- $\max \{\alpha_1\min\{ {D_j^+},  {D_j^-}\} + \alpha_2\max\{ {D_j^+},  {D_j^-}\}\}$\\
				to branch.

			\subsection{Branch on Important Variables First}
				Branch on variables that affects many decisions.

			\subsection{Some Search Strategy}
				- Best Bound First: select the node with the largest bound (good for closing the gap)\\
				- Deep First: Good for finding Lower bound and easier to do dual simplex\\
				- Mix: Start with \lq\lq{}Deep First\rq\rq{} until we find a good bound and do \lq\lq{}Best Bound First\rq\rq{}

		\section{Types of Branching}
			\subsection{Traditional Branching}
				For $\hat{x} \notin S$, $\exists j \in N$ such that
				\begin{equation}\hat{x_j} -\lfloor\hat{x_j}\rfloor > 0 \end{equation}
				Create two sub-problems
				\begin{figure}[!ht]
					\centering
					\begin{tikzpicture}[scale=0.2]
						\draw (0, 5) circle [radius=0.5];
						\draw (-5, 0) circle [radius=0.5];
						\draw (5, 0) circle [radius=0.5];
						\draw (0.353, 4.647) -- (4.647, 0.353);
						\draw (-0.353, 4.647) -- (-4.647, 0.353);
						\node [above] at (0, 5.5) {$\hat{x_j}$};
						\node [left] at (-2.5, 2.5) {$x_j \le \lfloor \hat{x_j} \rfloor$};
						\node [right] at (2.5, 2.5) {$x_j \ge \lceil \hat{x_j} \rceil$};
					\end{tikzpicture}
					\caption{Traditional Branching}
				\end{figure}

			\subsection{Constraint Branching}
				Use parallel constraints to branch, e.g.
				\begin{figure}[!ht]
					\centering
					\begin{tikzpicture}[scale=0.2]
						\draw (0, 5) circle [radius=0.5];
						\draw (-5, 0) circle [radius=0.5];
						\draw (5, 0) circle [radius=0.5];
						\draw (0.353, 4.647) -- (4.647, 0.353);
						\draw (-0.353, 4.647) -- (-4.647, 0.353);
						\node [above] at (0, 5.5) {$\hat{x}$};
						\node [left] at (-2.5, 2.5) {$x_1 - x_2 \ge 0$};
						\node [right] at (2.5, 2.5) {$-x_1 + x_2 \ge 1$};
					\end{tikzpicture}
					\caption{Traditional Branching}
				\end{figure}

			\subsection{SOS}
				For SOS1,
				\begin{figure}[!ht]
					\centering
					\begin{tikzpicture}[scale=0.2]
						\draw (0, 5) circle [radius=0.5];
						\draw (-5, 0) circle [radius=0.5];
						\draw (5, 0) circle [radius=0.5];
						\draw (0.353, 4.647) -- (4.647, 0.353);
						\draw (-0.353, 4.647) -- (-4.647, 0.353);
						\node [above] at (0, 5.5) {$\sum_{i=1}^{n}x_i=1$};
						\node [left] at (-2.5, 2.5) {$\sum_{i=1}^{k}x_i=1$};
						\node [right] at (2.5, 2.5) {$\sum_{i=k+1}^{n}x_i=1$};
					\end{tikzpicture}
					\caption{Traditional Branching}
				\end{figure}
				For SOS2 (using the first definition),
				\begin{figure}[!ht]
					\centering
					\begin{tikzpicture}[scale=0.2]
						\draw (0, 5) circle [radius=0.5];
						\draw (-5, 0) circle [radius=0.5];
						\draw (5, 0) circle [radius=0.5];
						\draw (0.353, 4.647) -- (4.647, 0.353);
						\draw (-0.353, 4.647) -- (-4.647, 0.353);
						\node [above] at (0, 5.5) {$\sum_{i=1}^{n}x_i \le 1$};
						\node [left] at (-2.5, 2.5) {$\sum_{i=1}^{k-1}x_i = 0$};
						\node [right] at (2.5, 2.5) {$\sum_{i=k+1}^{n}x_i =  0$};
					\end{tikzpicture}
					\caption{Traditional Branching}
				\end{figure}

			\subsection{GUB}
				This is where $x_i \in \{0, 1\}$, at most one variable can be 1,
				\begin{figure}[!ht]
					\centering
					\begin{tikzpicture}[scale=0.2]
						\draw (0, 5) circle [radius=0.5];
						\draw (-5, 0) circle [radius=0.5];
						\draw (5, 0) circle [radius=0.5];
						\draw (0.353, 4.647) -- (4.647, 0.353);
						\draw (-0.353, 4.647) -- (-4.647, 0.353);
						\node [above] at (0, 5.5) {$\sum_{i=1}^{n}x_i\le 1$};
						\node [left] at (-2.5, 2.5) {$\sum_{i=1}^{k}x_i=0$};
						\node [right] at (2.5, 2.5) {$\sum_{i=k+1}^{n}x_i=0$};
					\end{tikzpicture}
					\caption{Traditional Branching}
				\end{figure}

			\subsection{Ryan-Foster}
				Ryan-Foster is for Set covering problem. The typical model is
				\begin{align}
					\text{min} \quad & \sum_{i \in C} x_i \\
					\text{s.t.} \quad & \sum_{i \in C} a_{ij}x_{i} \ge 1, \quad \forall j \in U \\
							& x_i \in \{0, 1\}, \quad \forall i \in C 
				\end{align}
				\textbf{Observation} For any fractional solution, there are at least two elements $(i,j)$ so that $i$ and $j$ are both partially covered by the same set $S$, but there is another set $T$ that only covers $i$
				\begin{figure}[!ht]
					\centering
					\begin{tikzpicture}[scale=0.2]
						\draw (0, 5) circle [radius=0.5];
						\draw (-5, 0) circle [radius=0.5];
						\draw (5, 0) circle [radius=0.5];
						\draw (0.353, 4.647) -- (4.647, 0.353);
						\draw (-0.353, 4.647) -- (-4.647, 0.353);
						\node [left] at (-7, 2.5) {$(i,j)$ are};
						\node [left] at (-5, 0.5) { in the same set};
						\node [left] at (-4, 4.5) {if $a_{ik}a_{jk}=0 \Rightarrow x_k = 0$};
						\node [right] at (7, 2.5) {$(i,j)$ are};
						\node [right] at (5, 0.5) { in the different set};
						\node [right] at (3, 4.5) {if $a_{ik}=a_{jk}=1 \Rightarrow x_k=0$};
					\end{tikzpicture}
					\caption{Traditional Branching}
				\end{figure}

	\chapter{Branch and Cut}
		\section{Separation Algorithm}
			Basic idea is to separate the feasible region so that the current "solution" (which is an fractional solution) is not included in the feasible region.
			\subsection{Vertices Packing}
				The current solution is $\bar{x} \in [0,1]^n$, we have two options to do the separation:\\
				\textbf{Option 1 - find the maximum clique:}\\
				(This approach is as hard as the original problem)\\
				denote
				\begin{equation}
					y_i=\begin{cases}1, \text{ if } i \in C \\ 0, \text{ otherwise}\end{cases}
				\end{equation}
				Find the maximum clique via:
				\begin{align}
					\max \quad &\sum \bar{x_i} y_i  \\
					\text{s.t.} \quad & y_i + y_j \le 1, \forall \{i, j\} \notin E 
				\end{align}
				\textbf{Option 2 - Heuristic:}
				\begin{algorithm}[!ht]
					\caption{Heuristic method to find a clique}
					\begin{algorithmic}[1]
						\State find $v=\text{argmax}_{i\in V} \{\bar{x_i}\}, C=\{v\}$
						\While {$u\in \text{argmax}_{i \in \cap_{i \notin C}N_{(i, j)}\notin C} \{\bar {x_1}\}$ exists}
							\State C.add(u)
						\EndWhile
						\State return C
					\end{algorithmic}
				\end{algorithm}\\
				If $\sum_{i\in C} \bar{x_i} > 1$ then add cut $\sum_{i\in C} x_i \le 1$

			\subsection{TSP}
				When we have a solution, i.e. $\bar{x}$, perform the sub-tour searching algorithm, if there exists any sub-tour, add the corresponded constraint. That is the separation.

		\section{Optional v.s. Essential Inequalities}
			\subsection{Valid (Optional) Inequalities}
				See Figure \ref{OptInq}
				\begin{figure}[!ht]
					\centering
					\begin{tikzpicture}[node distance = 2cm]
						\node (LR) [circleNode] {LR};
						\node (CG) [process, below of=LR] {Cut Generation};
						\node (CLimit) [decision, below of=CG] {Generated?};
						\node (Cont) [process, below of=CG, xshift=4cm] {Continue};
						\node (NLR) [circleNode, below of=CLimit] {New LR};
						\node (LNLR) [circleNode, below of=NLR, xshift=-2cm] {Branch 1};
						\node (RNLR) [circleNode, below of=NLR, xshift=2cm] {Branch 2};
						\draw [arrow] (LR) -- (CG);
						\draw [arrow] (CG) -- (CLimit);
						\draw [arrow] (CLimit) -- node [right] {yes} (NLR);
						\draw [arrow] (CLimit) -- node [below] {no} (Cont);
						\draw [arrow] (Cont) |- (CG);
						\draw [arrow] (NLR) -- (LNLR);
						\draw [arrow] (NLR) -- (RNLR);
					\end{tikzpicture}
					\caption {Branch and Cut for Optional Inequality}\label{OptInq}
				\end{figure}

			\subsection{Essential Inequalities (Lazy Cuts)}
				See Figure \ref{EssInq}
				\begin{figure}[!ht]
					\centering
					\begin{tikzpicture}[node distance = 1.8cm]
						\node (IS) [circleNode, label = above:integer solution] {IS};
						\node (CG) [process, below of=IS] {Cut Generation};
						\node (CLimit) [decision, below of=CG] {Generated?};
						\node (FI) [circleNode, below of=CLimit, label = below:feasible integer solution] {FI};
						\node (LP) [process, below of=CG, xshift = 3.5cm] {Solve Linear Relaxation};
						\node (LPF) [decision, below of=LP] {Feasible?};
						\node (NLR) [circleNode, below of=LPF] {New LR};
						\node (LNLR) [circleNode, below of=NLR, xshift=-2cm] {Branch 1};
						\node (RNLR) [circleNode, below of=NLR, xshift=2cm] {Branch 2};
						\node (Cont) [process, below of=LP, xshift=2.5cm] {Continue};
						\draw [arrow] (IS) -- (CG);
						\draw [arrow] (CG) -- (CLimit);
						\draw [arrow] (CLimit) -- node [right] {no} (FI);
						\draw [arrow] (CLimit) -- node [below] {yes} (LP);
						\draw [arrow] (LP) -- (LPF);
						\draw [arrow] (LPF) -- node [right] {no} (NLR);
						\draw [arrow] (LPF) -- node [below] {yes} (Cont);
						\draw [arrow] (NLR) -- (LNLR);
						\draw [arrow] (NLR) -- (RNLR);
						\draw [arrow] (Cont) |- (CG);
					\end{tikzpicture}
					\caption {Branch and Cut for Essential Inequality}\label{EssInq}
				\end{figure}

		\section{Chvatal-Gomory Cut}
			\subsection{Chvatal-Gomory Rounding Procedure}
				For $x=P\cap \mathbb{Z}_+^n$, where $P=\{x\in \mathbb{R}_+^n|Ax \le b\}$, A is an mxn matrix with columns $\{a_1, ..., a_n\}$ and $u \in \mathbb{R}_+^n$\\
				- The inequality
				\begin{equation}
					\sum_{j=1}^n ua_jx_j\le ub 
				\end{equation}
				is valid\\
				- Therefore the inequality
				\begin{equation}
					\sum_{j=1}^n \lfloor ua_j \rfloor x_j \le ub 
				\end{equation}
				is valid\\
				- Furthermore, the inequality
				\begin{equation}
					\sum_{j=1}^n \lfloor ua_j \rfloor x_j \le \lfloor ub \rfloor 
				\end{equation}
				is valid.

			\subsection{Gomory Cutting Plane}
				For a IP problem
				\begin{align}
					\max \quad & cx  \\
					\text{s.t.} \quad & Ax=b  \\
						& x \in \mathbb{B}^n 
				\end{align}
				let $\bar{x}$ be an optimal basic solution for the LR of P.
				\begin{equation}
					\bar{x} = \left[\begin{matrix} B^{-1}b \\ 0 \end{matrix}\right] = \left[ \begin{matrix}x_B \\ x_N\end{matrix}\right] 
				\end{equation}
				We have
				\begin{align}
					& Bx_B + Nx_N = b \\
					\Rightarrow \quad & x_B + B^{-1}Nx_N=B^{-1}b  \\
					\Rightarrow \quad & x_B + [\bar{a}_1, \bar{a}_2, ...]x_N = \bar{b} \\
					\Rightarrow \quad & x_i + \sum_{j\in NB} \bar{a}_{ij}x_j = \bar{b}_i \quad \text{(for the $i$th row)} 
				\end{align}
				Assume that $x_i \in \{0, 1\}$, use CG-Procedure
				\begin{equation}
					x_i + \sum_{j \in NB} \lfloor \bar{a}_{ij} \rfloor x_j \le \lfloor \bar{b}_i \rfloor 
				\end{equation}
				is a valid constraint for $P$, furthermore,
				\begin{equation}
					(\bar{b}_i - \sum_{j\in NB} \bar{a}_{ij}x_j) + \sum_{j\in NB}\lfloor \bar{a}_{ij} \rfloor x_j\le \lfloor \bar{b}_i \rfloor 
				\end{equation}
				Move the item, we get a new Gomory Cutting Plane
				\begin{equation}
					\sum_{j\in NB} (\bar{a}_{ij} - \lfloor \bar{a}_{ij} \rfloor)x_j \ge \bar{b}_i - \lfloor \bar{b}_i \rfloor  
				\end{equation}
				Add this inequality to the LR, use the dual simplex method to do one pivot, we get a new solution. Use Gomory cutting plane iteratively and we can find the optimal solution for IP.

	\chapter{Packing and Matching}
		\section{Vertices Packing and Matching Formulation}
			Given a graph $G=(V, E)$, with $|V|=n$. A vertices packing solution is that no two neighboring vertices can be chosen at the same time.
			\begin{equation}
				PACK(G) = \{x\in \mathbb{B}^n|x_i + x_j \le 1, \forall (i, j)\in E\} 
			\end{equation}
			\begin{example}
				The following is an example:\\
				\begin{figure}[!ht]
					\centering
					\begin{tikzpicture}[node distance = 1.8cm]
						\node (1) [circleNode] {1};
						\node (2) [circleNode, below of=1] {2};
						\node (3) [circleNode, below of=2, xshift=-2cm] {3};
						\node (4) [circleNode, below of=2, xshift=2cm] {4};
						\draw (1) -- (2);
						\draw (1) -- (3);
						\draw (1) -- (4);
						\draw (2) -- (4);
						\draw (3) -- (4);
					\end{tikzpicture}
					\caption{Example of vertices packing problem}
				\end{figure}
				The PACK of this graph is\\
				\begin{equation}
					PACK = conv\left(
						\left(\begin{matrix}0 \\ 0 \\ 0 \\ 0\end{matrix}\right),
						\left(\begin{matrix}1 \\ 0 \\ 0 \\ 0\end{matrix}\right),
						\left(\begin{matrix}0 \\ 1 \\ 0 \\ 0\end{matrix}\right),
						\left(\begin{matrix}0 \\ 0 \\ 1 \\ 0\end{matrix}\right),
						\left(\begin{matrix}0 \\ 0 \\ 0 \\ 1\end{matrix}\right),
						\left(\begin{matrix}0 \\ 1 \\ 1 \\ 0\end{matrix}\right)
						\right)
				\end{equation}
			\end{example}

			Given a graph $G=(V, E)$, denote $\delta(i)$ as the set of all the edges introduced to vertice $i\in V$. A matching solution is that no two edges introduced to the same vertice can be chosen at the same time.
			\begin{equation}
				MATCH(G) = \{\sum_{e\in \delta(i)}x_e \le 1|i\in V\}
			\end{equation}

		\section{Dimension of PACK(G)}
			The dimension of PACK, i.e. $dim(PACK(G))$ is (full-dimensional)
			\begin{equation}
				dim(PACK(G)) = |V| 
			\end{equation}
			To prove that $dim(PACK(G)) = |V|$, we need to find $|V| + 1$ affinely independent vectors.\\
			\begin{proof}
				\begin{equation}
					rank\left(\left[\begin{matrix}0 & I_{|V|} \\ 1 & 1\end{matrix}\right]\right) = |V| + 1 
				\end{equation}
				Therefore, in PACK, $rank(A^=,b^=)=0$ 
			\end{proof}

		\section{Clique}
			- A \textbf{clique} is a subset of a graph that in the clique every two vertices linked with each other (complete sub-graph).
			- A \textbf{maximum clique} is a clique that any other vertice can not form a clique with all the points in this clique.

		\section{Inequalities and Facets of conv(VP)}
			\framebox{\textbf{Example:}}\\
				\begin{figure}[!ht]
					\centering
					\begin{tikzpicture}[node distance = 1.2cm]
						\node (1) [circleNode] {1};
						\node (2) [circleNode, below of=1, xshift=-2cm] {2};
						\node (3) [circleNode, below of=1, xshift=2cm] {3};
						\node (4) [circleNode, below of=2, xshift=2cm] {4};
						\node (5) [circleNode, below of=4, xshift=-0.9cm] {5};
						\node (6) [circleNode, below of=4, xshift=0.9cm] {6};
						\draw (1) -- (2);
						\draw (1) -- (3);
						\draw (1) -- (4);
						\draw (3) -- (4);
						\draw (2) -- (5);
						\draw (5) -- (6);
						\draw (6) -- (3);
					\end{tikzpicture}
					\caption{Example}
				\end{figure}

			\subsection{Type 1 (Nonnegative Constraints)}
				$x_i \ge 0$ induce facets.\\
				\framebox{\textbf{Proof:}}
				\begin{equation}
					rank\left(\left[\begin{matrix}0 & 0 \\ 0 & I_{|V|}\end{matrix}\right]\right) = |V| + 1 
				\end{equation}

			\subsection{Type 2 (Neighborhood Constraints)}
				$x_i + x_j \le 1$ is a valid constraint, but it \textbf{DOES NOT} always induce facet.

			\subsection{Type 3 (Odd Hole)}
				$H$ is an odd hole if it contains circle of $k$ nodes, such that $k$ is odd and there is no cords. e.g. \{1, 2, 5, 6, 3\}. Then, the following inequality is valid,
				\begin{equation}
					\sum_{i\in H}x_i\le \frac{|H|-1}2 
				\end{equation}
				Odd Hole inequality \textbf{DOES NOT} always induce facets.\\
				This inequality can be derived from Gomory cut.

			\subsection{Type 4 (Maximum Clique)}
				$C$ is a maximum clique, then the following inequality is valid and induce a facet,
				\begin{equation}
					\sum_{i\in C} x_i \le 1 
				\end{equation}
				\framebox{\textbf{Proof:}}\\
					First, if $C=V$
					\begin{equation}
						rank\left(\left[I\right]\right) = |C| = |V| 
					\end{equation}
					Second, if $C$ is a subset of $V$, for each vertice in $V \setminus C$, there should be at least one vertice in $C$ that is not linked with it. Therefore for each vertice in $C$ we can find a packing.

		\section{Gomory Cut in Set Covering}
			Consider a graph $G=(V, E)$, the covering problem is
			\begin{equation}
				\sum_{e\in \delta(i)}x_e \le 1, i\in V, x_e\in \{0, 1\}, e\in E
			\end{equation}
			For $T\subset V$, denote $\delta(i)$ as all edges induce to $i\in V$, denote $E(T) \subset E$ as all the edges linked between $(i, j), i\in T, j\in T$, therefore we have
			\begin{equation}
				\sum_{i\in T}\sum_{e\in \delta(i)}x_e \le |T| 
			\end{equation}
			For edges linking $i \in T, j \in T$, count them twice, for edges linking $i\in T, j\notin T$, count them once.We can have a new constraint
			\begin{equation}
				2\sum_{e\in E(T)}x_e + \sum_{e\in \delta(V\setminus T, T)}x_e \le |T| 
			\end{equation}
			Perform the Gomory Cut, the following constraint is a valid:
			\begin{equation}
				\sum_{e\in E(T)}x_e \le \lfloor \frac{|T|}2 \rfloor 
			\end{equation}

	\chapter{Traveling Salesman Problem}
		\section{TSP Formulation (Asymmetric)}
			Consider a Graph $G=\{A, N\}$\\
			Denote:\\
			\begin{equation}
				x_{ij} = \begin{cases}1, &\text{if goes from } i \text{ to } j\\ 0, & \text{otherwise}\end{cases}
			\end{equation}
			\textbf{Dantzig-Fulkerson-Johnson Formulation:}
			\begin{align}
				\min &\sum_{(i, j)\in A} c_{ij}x_{ij} \\
				& \sum_{j \in N, (i,j)\in A} x_{ij} = 1 \\
				& \sum_{i \in N, (i,j)\in A} x_{ij} = 1 \\
				& \sum_{j\notin S, i\in S, (i,j)\in A} x_{ij} = 1\text{ or } \sum_{i, j \in S, (i, j) \in A} x_{ij} \le |S| - 1 \\
				& \forall S \subset N, S\ne \emptyset, 2\le |S| \le n-1 
			\end{align}
			\textbf{Miller-Tucker-Zemlin Formulation:}
			\begin{align}
				\min &\sum_{(i, j)\in A} c_{ij}x_{ij} \\
				& \sum_{j \in N, (i,j)\in A} x_ij = 1 \\
				& \sum_{i \in N, (i,j)\in A} x_ij = 1 \\
				& u_i - u_j +nx_{ij}\le n-1 \quad i, j\in{2, ... , n}, (i, j)\in A \\
				& u_1 = 1 \\
				& 2 \le u_i \le n, i\in N, i>1 
			\end{align}

		\section{Sub-tour Searching Algorithm}
			In the graph $G=(N, A)$, let $\bar{G}=(N, \bar{A})$ be the connected components of graph, where
			\begin{equation}\bar{G}=(G, \bar{A}), \bar{A}=\{(i, j) \in A | \bar{x_{ij}}=1\} \end{equation}
			denote
			\begin{equation}\bar{FS}(i) = \{(i,j)\in \bar{A}\} \end{equation}
			Then the algorithm to find all sub-tour is the following:
			\begin{algorithm}[!ht]
				\caption{Sub-tour Searching Algorithm}
				\begin{algorithmic}[1]
					\State $K = \emptyset$
					\State $d_i = 0, \forall i \in N$
					\For {$i\in N$}
						\State $C = \emptyset$
						\State $Q = \emptyset$
						\If {$d_i == 0$}
							\State $d_i = 1$
							\State $C = C\cup \{i\}$
							\State Q.append(i)
							\While {$Q\ne \emptyset$}
								\State v = Q.pop()
								\For {$u \in \bar{FS}(v)$}
									\If {$d_u == 0$}
										\State $d_u = 1$
										\State $C = C \cup \{u\}$
										\State Q.append(u)
									\EndIf
								\EndFor
							\EndWhile
						\EndIf
						\State $K=K\cup C$
					\EndFor
				\end{algorithmic}
			\end{algorithm}

	\chapter{Knapsack Problem}
		\section{Knapsack Problem Formulation}
			Consider the knapsack set KNAP
			\begin{equation}conv(KNAP)= conv(\{x\in \mathbb{B}^n|\sum_{j\in N}a_jx_j\le b\})\end{equation}
			in where\\
			- $N = \{1, 2, ..., n\}$\\
			- With out lost of generality, assume that $a_j > 0, \forall j \in N$ and $a_j < b, \forall j \in N$

		\section{Valid Inequalities for a Relaxation}
			For $P=\{x\in \mathbb{B}^n | Ax\le b\}$, each row can be regard as a Knapsack problem, i.e. for row $i$
			\begin{equation}
				P_i = \{x\in \mathbb{B}^n | a_i^T x \le b_i\} 
			\end{equation}
			is a relaxation of $P$, therefore,
			\begin{equation}
				P\subseteq P_i, \forall i=1,2,...,m 
			\end{equation}
			\begin{equation}
				P\subseteq \cap_{i=1}^m P_i 
			\end{equation}
			So any inequality valid for a relaxation of an IP is also valid for IP itself.

		\section{Cover and Extended Cover}
			A set $C\subseteq N$ is a cover if $\sum_{j\in C} a_j > b$, a cover $C$ is minimal cover if
			\begin{equation}
				C\subseteq N | \sum_{j\in C}a_j>b, \sum_{j\in C\setminus k} a_j < b, \forall k \in C 
			\end{equation}
			For a cover $C$, we can have the cover inequality
			\begin{equation}
				\sum_{j\in C}x_j \le |C|-1
			\end{equation}
			The inequality is trivial considering the pigeonhole principle.\\
			$C\subseteq N$ is a minimal cover, then $E(C)$ is defined as following:
			\begin{equation}
				E(C) = C\cup \{j \in N | a_j \ge a_i, \forall i \in C\}
			\end{equation}
			is called an extended cover. Then we have,
			\begin{equation}
				\sum_{i\in E(C)} x_i \le |C| - 1 \text{ dominates } \sum_{i\in C} x_i \le |C| - 1
			\end{equation}
			and
			\begin{equation}
				\sum_{i\in E(C)} x_i \le |C| - 1 \text{ dominates } \sum_{i\in E(C)} x_i \le |E(C)| - 1
			\end{equation}
			Hereby we need to prove that $\sum_{i\in E(C)} x_i \le |C| - 1$ is valid, by contradiction.\\
			\framebox{\textbf{Proof:???}} Suppose $x^R \in KNAP$, $R$ is a feasible solution, Where
			\begin{equation}
				x^R_j = \begin{cases}1, \quad \text{if $j\in R$} \\ 0, \quad \text{otherwise}\end{cases} 
			\end{equation}
			Then
			\begin{equation}
				\sum_{j\in E(C)}x^R_j \ge |C| \Rightarrow |R \cap E(C)| \ge |C|  
			\end{equation}
			therefore
			\begin{equation}
				\sum_{j\in R}a_j \ge \sum_{j\in R \cap E(C)} a_j \ge \sum_{j\in C} a_j > b 
			\end{equation}
			which means $R$ is a cover, it is contradict to $\sum_{j\in E(C)}x^R_j \ge |C|$ so $x^R \notin KNAP$

		\section{Dimension of KNAP}
			$conv(KNAP)$ is full dimension, i.e. $dim(conv(KNAP))=n$.\\
			\framebox{\textbf{Proof:}} $0, e_j, \forall j\in N$ are $n + 1$ affinely independent points in $conv(KNAP)$\\

		\section{Inequalities and Facets of conv(KNAP)}
			\subsection{Type 1 (Lower Bound and Upper Bound Constraints):}
				- $x_k\ge 0$ is a facet of $conv(KNAP)$\\
				\framebox{\textbf{Proof:}} $0, e_j, \forall j\in N\setminus k$ are $n$ affinely independent points that satisfied $x_k=0$\\
				- $x_k\le 1$ is a facet iff $a_j + a_k \le b, \forall j\in N \setminus k$\\
				\framebox{\textbf{Proof:}} $e_k, e_j+e_k, \forall j \in N\setminus k$ are $n$ affinely independent points that satisfied $x_k = 1$

			\subsection{Type 2 (Extended Cover)}
				Order the variables so that $a_1 \ge a_2 \ge \dots \ge a_n$, therefore $a_1 = a_{max}$\\
				Let $C$ be a cover with $\{j_1, j_2, \dots, j_r\}$ where $j_1 < j_2 < \dots < j_r$ so that $a_{j_1} \ge a_{j_2} \ge \dots \ge a_{j_r}$\\
				Let $p = \min\{j | j\in N \setminus E(C)\}$\\
				Then
				\begin{equation}
					\sum_{j\in E(C)} x_j \le |C| - 1 
				\end{equation}
				is a facet of $conv(KNAP)$ if\\
				- $C = N$\\
				\framebox{\textbf{Proof:}}
				\begin{equation}
					R_k = C\setminus k, \forall k \in C = N \setminus k, \forall k \in N 
				\end{equation}
				have $|N|$ affinely independent vectors\\
				- $E(C) = N$ and $\sum_{j\in C \setminus \{j_1, j_2\}} a_j + a_{max} \le b$\\
				\framebox{\textbf{Proof:}} ($j_1, j_2$ are two heaviest elements in $C$)
				\begin{equation}
					S_k = C\setminus \{j_1, j_2\}\cup \{k\}, \forall k\in E(C)\setminus C 
				\end{equation}
				$R_k\cup S_k$ have $|C|+|E(C) \setminus C| = |E(C)| = |N|$ affinely independent vectors\\
				- $C = E(C)$ and $\sum_{j\in C \setminus j_1} a_j + a_p \le b$)\\
				\framebox{\textbf{Proof:}} ($j_1$ is the heaviest element in $C$, $k$ is the lightest element outside extended cover)
				\begin{equation}
					T_k = C \setminus j_i \cup \{k\}, \forall k\in N\setminus E(C) 
				\end{equation}
				$R_k \cup T_k$ have $|N \setminus E(C)| + |E(C)| = |N\setminus C| + |C| = |N|$ affinely independent vectors\\
				- $C \subset E(C) \subset N$ and $\sum_{j\in C \setminus \{j_1, j_2\}} a_j + a_{max} \le b$ and $\sum_{j\in C \setminus j_1} a_j + a_p \le b$\\
				\framebox{\textbf{Proof:}}$S_k \cup T_k$ have $|E(C) \setminus C| + |N \setminus E(C)| = |N|$ affinely independent vectors

		\section{Lifting}
			\subsection{Up Lifting}
				For KNAP problem
				\begin{equation}
					KNAP = \{x\in \mathbb{B}^n | \sum_j a_jx_j \le b\} 
				\end{equation}
				For $P=conv(KNAP)$ denote\\
				\begin{align}
					&P_{k_1, k_2, ..., k_m} \\
					&\quad =conv(KNAP\cap \{x\in \mathbb{B}^{n}|x_{k_1}=x_{k_2}=\dots=x_{k_m}=0\}) 
				\end{align}
				Therefore
				\begin{align}
					&P_{k_1, k_2, ..., k_m} \\
					&\quad=conv(KNAP\cap \{x\in \mathbb{B}^{n}|\sum_{j\in N\setminus \{k_1, k_2, ..., k_m\}} a_j x_j \le b\}) 
				\end{align}
				The $C=N$ cover inequality for $P_{k_1, k_2, ..., k_m}$ implies
				\begin{equation}
					\sum_{j\in N\setminus \{k_1, k_2, ..., k_m\}} x_j \le n-m-1 
				\end{equation}
				is a facet of $P_{k_1, k_2, ..., k_m}$\\
				The lifting process is to find a facet for $P_{k_1, k_2, ..., k_{m-1}}$ using facet of $P_{k_1, k_2, ..., k_m}$, i.e. find $\alpha_{m}$ for the following constraint to be a facet.
				\begin{equation}
					\alpha_m x_m + \sum_{j\in N\setminus \{k_1, k_2, ..., k_m\}} x_j \le n-m-1 
				\end{equation}
				If $x_m=0$, $\alpha_m \ge 0$,\\
				If $x_m=1$, $\alpha_m \le (n-m+1) - \gamma$ where
				\begin{align}
					\gamma &= \max\{\sum_{j\in N\setminus\{k_1, k_2, ..., k_m\}}x_j|x\in P_{k_1, k_2, ..., k_{m-1}}, x_m=1\} \\
					&= \max\{\sum_{j\in N\setminus\{k_1, k_2, ..., k_m\}}x_j|\sum_{j \in N \setminus \{k_1, k_2, ..., k_m\}}a_jx_j\le b-a_m\} 
				\end{align}
				Then let $\alpha_m = n-m+1-\gamma$, we uplifted a constraint. Repeat this procedure for $\{k_1, k_2, ..., k_m\}$ and finally we can find a family of facets for $conv(KNAP)$


			\subsection{Down Lifting}
				Similar to up lifting, we can perform the lifting in a different way.\\
				Denote
				\begin{align}
					&P_{k_1, k_2, ..., k_m}^{'} \\
					&\quad =conv(KNAP\cap \{x\in \mathbb{B}^{n}|x_{k_1}=x_{k_2}=\dots=x_{k_m}=1\}) 
				\end{align}

		\section{Separation of a Cover Inequality}
			$C\in N$ is a cover if $\sum_{i\in C} a_i > b$, let $C$ be a minimal cover
			\begin{align}
				&\sum_{i\in C}x_i \le |C| - 1 \\
				\Rightarrow \quad & |C| - \sum_{i\in C}x_i = \sum_{i \in C}(1-x_i)\ge 1 \\
			\end{align}
			let $\bar{x}$ be a fractional solution of $\{\sum_{i\in N} a_ix_i \le b, x_i \in [0, 1], i\in N\}$, find a cover $C$ of which $\sum_{i\in C}(1-\bar{x_i})<1$\\
			Decision variable:
			\begin{equation}
				y_i = \begin{cases}1, \quad \text{if } i\in C\\ 0, \quad \text{otherwise}\end{cases}
			\end{equation}
			\begin{align}
				\min \quad & \sum_{i\in N} (1-\bar{x_i})y_i = z \\
				\text{s.t.} \quad & \sum_{i\in N}a_iy_i \ge b+1 \\
				&y_i \in \{0, 1\}, i\in N
			\end{align}
			if $z<1$, then the cover cut associated with $y$ is violation by $\bar{x}$

	\chapter{Network Flow Problem}
		(Network Flow Problem is a special type of IP problem, its linear relaxation is the convex hull of the original problem.)
		\section{Shortest Path Problem}
			A graph $G=(A, N)$ is a directed graph\\
			\begin{figure}[!ht]
				\centering
				\begin{tikzpicture}[node distance = 1.8cm]
					\node (1) [circleNode, label=above:start] {1};
					\node (2) [circleNode, right of=1, yshift=1cm] {2};
					\node (3) [circleNode, right of=1, yshift=-1cm] {3};
					\node (4) [circleNode, right of=2] {4};
					\node (5) [circleNode, right of=3] {5};
					\node (6) [circleNode, label=above:end, right of=4, yshift=-1cm] {6};
					\draw [arrow] (1) -- (2);
					\draw [arrow] (1) -- (3);
					\draw [arrow] (2) -- (3);
					\draw [arrow] (2) -- (4);
					\draw [arrow] (3) -- (5);
					\draw [arrow] (4) -- (6);
					\draw [arrow] (5) -- (6);
					\draw [arrow] (5) -- (4);
					\draw [arrow] (2) -- (5);
				\end{tikzpicture}
				\caption{Example of directed graph}
			\end{figure}
			Denote:\\
			\begin{equation}
				x_{ij} = \begin{cases}1, &\text{if goes from } i \text{ to } j\\ 0, & \text{otherwise}\end{cases}
			\end{equation}
			The shortest path problem can be formulated as the following:\\
			\begin{align}
				\min &\sum_{(i, j)\in A} c_{ij}x_{ij} \\
				& \sum_{i \in N\setminus(\{S\}\cup\{E\}), (i,j)\in A} x_{ij} - \sum_{j \in N, (i,j)\in A} x_{ji} = 0 \\
				& \sum_{i=\{S\}, (i,j)\in A} x_{ij} - \sum_{j \in N, (i,j)\in A} x_{ji} = 1 \\
				& \sum_{i=\{E\}, (i,j)\in A} x_{ij} - \sum_{j \in N, (i,j)\in A} x_{ji} = -1 \\
				& x_{ij} \in [0,1], (i,j)\in A 
			\end{align}
			Although initially $x_{ij} \in [0,1]$, in the optimized solution, $x\in \{0, 1\}$.

		\section{Maximum Flow Problem}
			\begin{align}
				\min &\sum_{(i, j)\in A} F \\
				& \sum_{i \in N\setminus(\{S\}\cup\{E\}), (i,j)\in A} x_{ij} - \sum_{j \in N, (i,j)\in A} x_{ji} = 0 \\
				& \sum_{i=\{S\}, (i,j)\in A} x_{ij} - \sum_{j \in N, (i,j)\in A} x_{ji} = F \\
				& \sum_{i=\{E\}, (i,j)\in A} x_{ij} - \sum_{j \in N, (i,j)\in A} x_{ji} = -F \\
				& l_{ij} \le x_{ij} \le u_{ij}, (i,j)\in A 
			\end{align}
			In where $F$ means the flow from source to target.

		\section{Minimum Cost Flow}
			The shortest path problem is a special case of Minimum Cost Flow Problem, which can be formulated as the following:\\
			\begin{align}
				\min &\sum_{(i, j)\in A} c_{ij}x_{ij} \\
				& \sum_{i \in N\setminus(\{S\}\cup\{E\}), (i,j)\in A} x_{ij} - \sum_{j \in N, (i,j)\in A} x_{ji} = 0 \\
				& \sum_{i=\{S\}, (i,j)\in A} x_{ij} - \sum_{j \in N, (i,j)\in A} x_{ji} = 1 \\
				& \sum_{i=\{E\}, (i,j)\in A} x_{ij} - \sum_{j \in N, (i,j)\in A} x_{ji} = -1 \\
				& x_{ij} \in [0,1], (i,j)\in A 
			\end{align}

		\section{Unimodularity}
			\subsection{Unimodular Matrix and Total Unimodular Matrix}
				- A unimodular matrix $M$ is a squared matrix, where $det(M)=1$ or $-1$.\\
				- Total unimodular matrix is a matrix where all its sub-matrices are unimodular matrix.

			\subsection{Importance of Unimodular Matrix}
				Let $M_{m\times m}$ be a unimodular matrix, if $b\in \mathbb{Z}^m$, the solution for $Mx=b$ is always integer.\\
				\framebox{\textbf{Proof:}} By Cramer's Rule
				\begin{equation}
					x_i = \frac{det{M_i}}{det{M}} 
				\end{equation}
				in which $M_i$ is matrix $M$ replace $i$th column with $b$. Therefore $det(M_i)$ is integer. Also, $det(M)\ne 0$, so $det(M)=1$ or $det(M)=-1$. Proved.

			\subsection{Structures of Total Unimodular Matrix}
				\textbf{Structure 1:}
					Matrix $M$ that has only 1, -1, 0 enters and each column has at most 2 non-zeros is a TU matrix if it satisfies the following conditions:\\
					We can split the rows in to tops and bottoms, such that for all columns $j$ having 2 non-zeros\\
					- If the non-zeros have the same sign, then one value should be in top and the other should be in bottom\\
					- If the non-zeros have the different sign, then both of them should be in top or both of them should be in bottom\\
				\textbf{Structure 2:}
					If all the columns in matrix $M$ has only 0 or consecutive 1s (or -1s), matrix $M$ is a TU matrix

			\subsection{Construct a New Unimodular Matrix}
				Let $F$ be a unimodular matrix, then
				\begin{equation}
					\left[\begin{matrix}F \\ I\end{matrix}\right] 
				\end{equation}
				is a unimodular matrix, also
				\begin{equation}
					\left[\begin{matrix}F & 0 \\ I & I\end{matrix}\right] 
				\end{equation}
				is a unimodular matrix.
\end{document}